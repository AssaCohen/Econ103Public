\documentclass[12pt,letterpaper]{article}
\usepackage[utf8]{inputenc}
\usepackage{amsmath}
\usepackage{amsfonts}
\usepackage{amssymb}
\usepackage{graphicx}
\usepackage[margin=1in]{geometry}

\title{How to Do Well in Econ 103}
\author{Francis J.\ DiTraglia}
\date{}
\begin{document}

\maketitle

% \section*{Introduction}
% There's no way around it: Statistics for Economists (Econ 103) is one of the hardest required courses in the economics major. Students who are used to doing well in their other coursework are often confused when they find themselves falling behind in Econ 103. This note is an outgrowth of the advice I've given such students in past semesters when they've come to office hours for advice on how best to study for the course.  

\section{Econ 103 is un-crammable: clock in every week.}
It's possible to get an A in some courses by cramming for 48 hours before each exam. Here I speak from experience. Econ 103 is different. If you try to cram for this course, I'm willing to bet that you'll be sending me a frantic email once final grades have been posted. This is not because I'm a particularly harsh grader. (In fact, I use the most generous grading curve permitted by the department.) It is simply the nature of the course material. If you've ever tried to learn a foreign language or a musical instrument, or trained for competitive sports, you'll know what I mean. It's impossible to learn the piano, become fluent in Spanish, or train for a marathon by spending 18 hours a day over a three-day weekend. 

The analogy to learning a musical instrument is a particularly good one. There are certain facts and definitions in Econ 103 that you'll need to memorize, just as you need to memorize basic musical notation and the locations of the notes on the keyboard if you want to learn the piano. \emph{Merely} memorizing the basics, however, will no more ensure your success in this course than it will make you a concert pianist. Only regular practice \emph{applying} what you know can do that. If you wanted to learn the piano, I'd advise you to spend an hour each day. It's not essential that you study for Econ 103 every day, but you should expect to spend at least seven hours each week studying for this course, not counting time spent in lectures and recitations. If you don't put in the hours, you can't expect to do well.


\section{Think; don't try to learn by rote.}
A few years ago I read about an experiment in which subjects were asked to memorize a randomly-chosen configuration of pieces on a chessboard. The idea was to test whether chess grand masters performed better than ordinary people like you or me. The results were fascinating. When the chessboard was initialized \emph{completely at random}, with no regard for whether the arrangement of pieces could occur in a real game of chess, grand masters and ordinary people did just as well. When the arrangement of pieces was \emph{constrained} to obey the rules of chess, however, grand masters had essentially perfect recall whereas ordinary people did no better than before. What this demonstrates is both subtle and important. Chess grand masters \emph{are} much better at memorizing a chessboard than you or me. Their advantage, however, it tied to a \emph{contextual knowledge} of the game of chess. Using this advantage, grand masters are effectively able to \emph{compress} the information contained in a chessboard, rather than memorizing the location of each piece separately. This is exactly how you should approach Econ 103.

A student once told me that the number of formulas he was expected to know for the second midterm in this course was ``completely unreasonable.'' Indeed, he considered it ``absurd'' that I didn't permit formula sheets during the exam. Impoliteness aside, in an important sense I agree with this student's assessment. If you make a list of all the formulas used in Econ 103, you'll find that it is extremely long. Trying to memorize it directly, with no context, would be just as hard as memorizing a chessboard piece-by-piece. But like the pieces in a game of chess, all of the formulas in the course are linked by a relatively small number of rules. The list this student was trying to memorize only \emph{appeared} long to him because he didn't understand these rules. Attempting to learn by rote won't get you very far in this course: it's essential to think carefully about the \emph{concepts}.

Students often ask me which homework questions they should solve to ensure that they'll do well on the exam. When I tell them that no such set of problems exists, their response is typically shock and dismay: ``I thought you said the homework was important!'' This is a difficult point. The homework is extremely important (see below), but the \emph{particular questions} are far less important than the concepts they illustrate. This is perhaps different from what you're used to from your introductory calculus course, where the emphasis was on practicing a small number of skills. Econ 103 will furnish you with many important skills and the homework will give you an opportunity to practice them. To pass the exams, you'll most certainly need to know these skills. To do \emph{well} on the exams, however, you'll need to demonstrate that you've learned how to \emph{think} statistically. \emph{Understanding} uncertainty is hard enough, let alone trying to \emph{quantify} it. In this course we'll do both. This won't be easy, but I hope to convince you over the course of the semester that it's worth the effort. For now, take my word for it. When you're working on the homework problems and R tutorials, get in the habit of asking yourself how this material relates to what you've learned to far, and focus on \emph{concepts} as well as skills.


\section{Go Through the Slides Carefully After Each Lecture}
While attending lectures is important, it's not enough. 
The only way to make sure that you don't fall behind in this course is to go through all the slides carefully after each lecture.
(I'll post these online along with the lecture recordings.)
This course is incredibly cumulative: ideas that we cover in the first few weeks of the semester re-appear again and again.
If you don't come to a lecture with a good understanding of what we've covered so far, you'll only compound the problem, making it even harder to get caught up.
I've seen it happen to many students in the past, and I don't want it to happen to you.

So how should you go about reviewing the slides?
The first step is to make sure you understand all the \emph{basic definitions} and \emph{terminology} introduced in the lecture.
This may seem obvious, but it's crucial: unless you first learn the \emph{language} of probability and statistics, you'll never be able to understand the \emph{ideas}.
I've seen many students struggle with this in the past.
Once you have a good handle on the definitions, the next step is to make a list of the key \emph{concepts} introduced in the lecture. 
You may not completely understand them at this point, but that's ok.
The point is to make sure you can articulate what it is that I'm trying to get you to understand in a given lecture.
Once you know the definitions and have identified the key concepts, I suggest that you go back to the beginning of the lecture and work through everything line-by-line.
If there's anything you don't understand, watch the relevant part of the lecture recording to see if this helps.
If it doesn't, circle what you don't understand and move on.
When you reach the end of the slides, make a list of all the things you circled, with a brief description of what you're confused about.
Expect to be confused by this material at first: it's hard!
If you don't have any questions after working through a lecture, chances are you're not really grappling with the material.

Now you're in a great position.
You have two lists: the first tells you exactly what the \emph{point} of this lecture was, and the second one tells you exactly what you don't understand.
The first list will help you to focus on the ``big picture'' and the second one will help you to fill in the gaps in your knowledge.
To get your questions answered, the best place to go is Piazza.
Before you post, browse through the questions that have already been posted.
If you have a question, chances are someone else does too and it may already have been answered.
If there's a similar question that doesn't quite explain what you need to know, try adding a follow-up.
This keeps related information in the same place, making it easier to find answers on Piazza.
It also shows me and the RIs that more than one person is confused about a certain issue, so we know it's worth spending time on.
If there's no post that answers your question and no post similar enough for you to post a follow-up, create a new question.
To increase the chance of getting a good answer \emph{be as clear and detailed as possible}.
Make sure to explain exactly which lecture and slide you're confused about, otherwise we may not know what you're referring to.
``I don't understand statistical power'' is an example of a \emph{bad question} -- as written it doesn't have an answer.
To turn this into a good question that is easy to answer, all you need to do is be clear about what you \emph{do} and \emph{do not} understand about statistical power.
Office hours and recitations are also a good place to ask questions.
The main advantage of Piazza, however, is that it creates a record of every answer.
This is a ``positive externality'' and something that the RIs and I definitely want to encourage.

\section{Quizzes Cover the Basics to Keep You On Track}
There will be a number of quizzes in recitation over the course of the semester.
(See the syllabus for details.)
These will be \emph{short} and \emph{easy}.
The idea is to make sure you understand the \emph{basic material} from the lectures covered since the last quiz or exam -- definitions, formulas, and simple examples -- and reward you for staying on track.
In particular, if you spend a couple of hours reviewing the slides you should expect to be able to get at least a ``B'' and very possibly an ``A'' on that week's quiz.
Because they are intended to be easy, quizzes are \emph{not} a good guide to wehether you are well prepared for an exam.
If you consistently do well on the quizzes, you definitely know the basics.
Exams, however, will go \emph{beyond} the basics.
If you want to do well on them, the best measure of your understanding is the homework and past exams.


\section{Do the homework.}
Homework will consist of problem sets and R Tutorials which I'll post on Piazza. 
Since R Tutorials are largely separate from the lectures, you can really do them at any time: they could be a good break if you're getting stuck on the lecture slides.
Problems sets are a combination of odd-numbered problems from the book and more challenging ``Additional Problems'' drawn from past exams and other sources. 
The questions from the book tend to be easier and I've assigned a lot of them so you have plenty of practice material.
If you find the questions from the book too easy, you don't necessarily have to solve them all.
In general, the ``Additional Problems'' are more important.
It's probably best to wait until you feel you have a good understanding of the week's material before attempting the problem set. 
Otherwise you're likely to get frustrated and waste time.

Homework is neither collected nor graded but it is extremely important that you work through it yourself. 
Much of the learning you'll do in this class takes place at home, not in the lectures.
Homework is a crucial part of this.
Working through difficult questions both deepens your understanding and points out gaps in your knowlege that you might otherwise have missed.
This is why it is \emph{absolutely essential} that you make a serious attempt to solve a problem before looking at the solution.
If you're stuck, take a break and come back to it.
Try to get a hint from Piazza or a classmate.
The solutions are there for you to \emph{check your work} and as a \emph{last resort} if you're really stuck.
Don't fall into the trap of thinking that you can simply read through the solutions rather than working the problems yourself.
If you do, you'll be in for an unpleasant surprise when midterms roll around.

\section{Yes, you really do need to learn R.}
R is an integral part of this course. To my mind there's no point teaching the theory of statistics without also giving students the tools to \emph{use} statistics. You'll learn to use R in a series of tutorials assigned as homework. Later in the semester, R material will be integrated directly into homework problems, giving us the opportunity to look at much more interesting questions than you can handle with a pencil and paper.  Please be aware that I \emph{will} ask questions about R on exams. It's fairly clear, however, that a great deal of what's included in the R assignments can't easily be tested in-class. It would be a mistake, however, to conclude that you don't need to really \emph{do} the R assignments. First, despite the limitations of in-class assessment, it's actually quite easy to write questions that distinguish between who actually did the assignments versus who merely \emph{read} them. Second, and more importantly, the R material is carefully designed to strengthen your understanding of the concepts covered in class. The very best way to ensure that you understand a statistical procedure is to write computer code to implement it. I can't tell you how many times I've found myself flummoxed when trying to code up a procedure I thought I knew inside-out. Invariably, getting it to work on the computer has led me to a much deeper understanding of the theory. The R material can be tricky, so you should expect to be confused at first. Piazza is a great place to get hints and pointers if you find yourself stuck. Just be careful not to fall into the trap of merely reading someone else's solution and assuming you understand the assignment.

 






\end{document}
