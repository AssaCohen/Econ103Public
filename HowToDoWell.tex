\documentclass[12pt,letterpaper]{article}
\usepackage[utf8]{inputenc}
\usepackage{amsmath}
\usepackage{amsfonts}
\usepackage{amssymb}
\usepackage{graphicx}
\usepackage[margin=1in]{geometry}

\title{How to Do Well in Econ 103}
\author{Francis J.\ DiTraglia}
\date{}
\begin{document}

\maketitle

\section*{Introduction}
There's no way around it: Statistics for Economists (Econ 103) is one of the hardest required courses in the economics major. Students who are used to doing well in their other coursework are often confused when they find themselves falling behind in Econ 103. This note is an outgrowth of the advice I've given such students in past semesters when they've come to office hours for advice on how best to study for the course.  

\section{Econ 103 is un-crammable: clock in every week.}
It's possible to get an A in some courses by cramming for 48 hours before each exam. Here I speak from experience. Econ 103 is different. If you try to cram for this course, I'm willing to bet that you'll be sending me a frantic email once final grades have been posted. This is not because I'm a particularly harsh grader. (In fact, I use the most generous grading curve permitted by the department.) It is simply the nature of the course material. If you've ever tried to learn a foreign language or a musical instrument, or trained for competitive sports, you'll know what I mean. It's impossible to learn the piano, become fluent in Spanish, or train for a marathon by spending 18 hours a day over a three-day weekend. 

The analogy to learning a musical instrument is a particularly good one. There are certain facts and definitions in Econ 103 that you'll need to memorize, just as you need to memorize basic musical notation and the locations of the notes on the keyboard if you want to learn the piano. \emph{Merely} memorizing the basics, however, will no more ensure your success in this course than it will make you a concert pianist. Only regular practice \emph{applying} what you know can do that. If you wanted to learn the piano, I'd advise you to spend an hour each day. It's not essential that you study for Econ 103 every day, but you should expect to spend at least seven hours each week studying for this course. If you don't put in the hours, you can't expect to do well.


\section{Think; don't try to learn by rote.}
A few years ago I read about an experiment in which subjects were asked to memorize a randomly-chosen configuration of pieces on a chessboard. The idea was to test whether chess grand masters performed better than ordinary people like you or me. The results were fascinating. When the chessboard was initialized \emph{completely at random}, with no regard for whether the arrangement of pieces could occur in a real game of chess, grand masters and ordinary people did just as well. When the arrangement of pieces was \emph{constrained} to obey the rules of chess, however, grand masters had essentially perfect recall whereas ordinary people did no better than before. What this demonstrates is both subtle and important. Chess grand masters \emph{are} much better at memorizing a chessboard than you or me. Their advantage, however, it tied to a \emph{contextual knowledge} of the game of chess. Using this advantage, grand masters are effectively able to \emph{compress} the information contained in a chessboard, rather than memorizing the location of each piece separately. This is exactly how you should approach Econ 103.

A student once told me that the number of formulas he was expected to know for the second midterm in this course was ``completely unreasonable.'' Indeed, he considered it ``absurd'' that I didn't permit formula sheets during the exam. Impoliteness aside, in an important sense I agree with this student's assessment. If you make a list of all the formulas used in Econ 103, you'll find that it is extremely long. Trying to memorize it directly, with no context, would be just as hard as memorizing a chessboard piece-by-piece. But like the pieces in a game of chess, all of the formulas in the course are linked by a relatively small number of rules. The list this student was trying to memorize only \emph{appeared} long to him because he didn't understand these rules. Attempting to learn by rote won't get you very far in this course: it's essential to think carefully about the \emph{concepts}.

Students often ask me which homework questions they should solve to ensure that they'll do well on the exam. When I tell them that no such set of problems exists, their response is typically shock and dismay: ``I thought you said the homework was important!'' This is a difficult point. The homework is extremely important (see below), but the \emph{particular questions} are far less important than the concepts they illustrate. This is perhaps different from what you're used to from your introductory calculus course, where the emphasis was on practicing a small number of skills. Econ 103 will furnish you with many important skills and the homework will give you an opportunity to practice them. To pass the exams, you'll most certainly need to know these skills. To do \emph{well} on the exams, however, you'll need to demonstrate that you've learned how to \emph{think} statistically. \emph{Understanding} uncertainty is hard enough, let alone trying to \emph{quantify} it. In this course we'll do both. This won't be easy, but I hope to convince you over the course of the semester that it's worth the effort. For now, take my word for it. When you're working on the homework problems and R tutorials, get in the habit of asking yourself how this material relates to what you've learned to far, and focus on \emph{concepts} as well as skills.


\section{Read the text before class and the slides after.}
I spent a lot of time choosing a textbook for this course and selecting readings. It's not a perfect fit for what I cover in lectures but it provides extremely good intuition for the key concepts. Accordingly, I strongly recommend that you do all of the readings for the course. (I'll post a list of readings on Piazza.) When you do the readings, try to focus on the concepts. Once you understand these, the details will be much easier.

 Please be aware that you are only responsible for material covered in the lecture slides, homework and R tutorials. If you see something in the text that never appears elsewhere, don't worry about it. Ideally, you should do the readings from the textbook before class and then go over the lecture slides after class to make sure that you understand everything. If you don't, this is an excellent opportunity to ask a question on Piazza before attempting the homework. 


\section{Do the homework.}
Homework will consist of problem sets and R Tutorials whilch I'll post on Piazza at the beginning of the semester. Problems sets are a combination of
odd-numbered problems from the book and more challenging examples of my own based on past exam problems. This homework is neither collected nor graded but it is extremely important that you work through it yourself. Merely reading the answers only creates the illusion of learning. Remember point \#3 from above when you're doing the homework: \emph{think; don't try to learn by rote}.

\section{Yes, you really do need to learn R.}
R is an integral part of this course. To my mind there's no point teaching the theory of statistics without also giving students the tools to \emph{use} statistics. You'll learn to use R in a series of tutorials assigned as homework. Later in the semester, R material will be integrated directly into homework problems, giving us the opportunity to look at much more interesting questions than you can handle with a pencil and paper.  Please be aware that I \emph{will} ask questions about R on exams. It's fairly clear, however, that a great deal of what's included in the R assignments can't easily be tested in-class. It would be a mistake, however, to conclude that you don't need to really \emph{do} the R assignments. First, despite the limitations of in-class assessment, it's actually quite easy to write questions that distinguish between who actually did the assignments versus who merely \emph{read} them. Second, and more importantly, the R material is carefully designed to strengthen your understanding of the concepts covered in class. The very best way to ensure that you understand a statistical procedure is to write computer code to implement it. I can't tell you how many times I've found myself flummoxed when trying to code up a procedure I thought I knew inside-out. Invariably, getting it to work on the computer has led me to a much deeper understanding of the theory. The R material can be tricky, so you should expect to be confused at first. Piazza is a great place to get hints and pointers if you find yourself stuck. Just be careful not to fall into the trap of merely reading someone else's solution and assuming you understand the assignment.

 






\end{document}