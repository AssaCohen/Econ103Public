\documentclass{article}

\usepackage{amsfonts}

\begin{document}

\title{Recitation 3}

\maketitle

\section{Refugee Screening}

There are about 7 billion people in the world. About 23 million of them live in Syria.

Let's say there are 200,000 terrorists in the world, 30,000 of whom are in Syria.

Lastly, the United States has admitted about 895,055 refugees in the past 15 years, 3 of whom would eventually commit an act of terror, compared to roughly 70 of roughly 130 million US citizens (excluding children and people over 50) who did the same.

\begin{enumerate}
\item Given these rough estimates, what is the probability that a randomly selected person in the world is a terrorist?
\item Given that someone is a terrorist, what is the probability that they are Syrian?
\item You're screening refugees and hoping to weed out any terrorist posing as a refugee. Given that you know the refugee is Syrian, what is the probability they are a terrorist? That is, what is $\mathbb{P}[terrorist|Syrian]$?
\item By what order of magnitude does knowing a person is Syrian increase the likelihood that they're a terrorist?
\item What is the probability that, given someone is a refugee in the US, they are a terrorist?
\item What is the probability that, given someone is \textit{not} a refugee in the US, they are a terrorist?
\item By what order of magnitude does knowing a person is a refugee increase the likelihood that they're a terrorist?
\item \textit{(open-ended)} This is of course a simplistic interpretation of the problem of refugee screening. Why does the simplistic analysis break down? What are other important considerations? What assumptions are violated?
\end{enumerate}

\end{document}