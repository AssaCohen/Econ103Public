\documentclass[addpoints,12pt]{exam}
\usepackage{amsmath, amssymb}
\linespread{1.1}
\usepackage{hyperref}
\usepackage{enumerate}
\usepackage{multirow}

%To include the answers use:
%	pdflatex "\def\showanswers{1} \input{thisfile.tex}"
\ifdefined\showanswers
  \printanswers
\else
  \noprintanswers
\fi

\title{Problem Set \#6}
\author{Econ 103}
\date{}
\begin{document}
\maketitle

% This problem set concerns continuous random variables

\section*{Part I -- Problems from the Textbook}
Chapter 4: 19, 21, 23
(\emph{When necessary, use R rather than the Normal tables in the front of the textbook.})

% \section*{Part II -- R Tutorial}
% Complete ``R Tutorial \#5'' available at:\\
% \url{http://www.ditraglia.com/econ103/Rtutorial5.html}.


\section*{Part II -- Additional Problems}


\begin{questions}


\question Suppose that $X$ is a random variable with the following PDF
	$$f(x)=\left\{\begin{array}{ll}x & 0\leq x\leq1 \\ 2-x  & 1 \leq x \leq 2\\ 0  & \mbox{otherwise}\end{array}\right. $$	
	\begin{parts}
		\part Graph the PDF of $X$.
			\begin{solution}
				It's an isosceles triangle with base from (0,0) to (2,0) and height 1. 
			\end{solution}
		\part Show that $\int_{-\infty}^\infty f(x)\; dx = 1$.
			\begin{solution}
				\begin{eqnarray*}
				\int_{-\infty}^\infty f(x)\; dx &=& \int_0^1 x\; dx + \int_1^2 (2-x)\; dx = \left.\frac{x^2}{2} \right|_0^1 + \left. \left(2x - \frac{x^2}{2}\right) \right|_1^2 \\ \\
					&=& 1/2 +( 4 - 2) - (2 - 1/2) = 1
				\end{eqnarray*}
			\end{solution}
		\part What is $P(0.5 < X<1.5)$?
			\begin{solution}
				\begin{eqnarray*}
					P(0.5 < X<1.5) &=& \int_{0.5}^{1.5} f(x) \; dx = \int_{0.5}^1 x\; dx + \int_1^{1.5} (2-x)\; dx\\\\
						&=& \left.\frac{x^2}{2} \right|_{0.5}^1 + \left. \left(2x - \frac{x^2}{2}\right) \right|_1^{1.5}\\ \\
			& =& (1/2 - 1/8) + (3 - 9/8) - (2  - 1/2) \\
			&=& 3/8 + 15/8 - 2 + 1/2 =  18/8 - 16/8  + 4/8\\ 
			&=& 6/8 = 3/4 = 0.75
				\end{eqnarray*}
			\end{solution}
	\end{parts}
	
	

\question A random variable is said to follow a Uniform$(a,b)$ distribution if it is equally likely to take on any value in the range $[a,b]$ and never takes a value outside this range. Suppose that $X$ is such a random variable, i.e.\ $X \sim \mbox{Uniform}(a,b)$.
	\begin{parts}
		\part What is the support of $X$?
			\begin{solution}
				$[a,b]$
			\end{solution}
		\part Explain why the PDF of $X$ is $f(x) = 1/(b-a)$ for $a\leq x \leq b$, zero elsewhere.
			\begin{solution}
				This simply generalizes the Uniform$(0,1)$ random variable from class. To capture the idea that $X$ is equally likely to take on any value in the range $[a,b]$, the PDF must be constant. To ensure that it integrates to 1, the denominator must be $b-a$. 
			\end{solution}
		\part Using the PDF from part (b), calculate the CDF of $X$.
		\begin{solution}
		 \begin{eqnarray*}
		 	F(x_0) &=&\int_{-\infty}^{x_0} f(x)\; dx = \int_{a}^{x_0} \frac{dx}{b-a} = \left.\frac{x}{b-a}\right|_a^{x_0} =  \frac{x_0-a}{b-a} 
		\end{eqnarray*}
		 \end{solution}
		\part Verify that $f(x) = F'(x)$ for the present example.	
		\begin{solution}
			\begin{eqnarray*}
				F'(x) = \frac{d}{dx} \left(\frac{x - a}{b-a}\right) = \frac{1}{b-a} = f(x)
			\end{eqnarray*}
		\end{solution}
		\part Calculate $E[X]$.
			\begin{solution}
			\begin{eqnarray*}
				E[X] &=& \int_{-\infty}^\infty xf(x)\; dx = \int_a^b \frac{x}{b-a}\; dx = \left.\frac{x^2}{2(b-a)}\right|_a^b =  \frac{b^2 - a^2}{2(b-a)} = \frac{a+b}{2}
			\end{eqnarray*}
			\end{solution}
		\part Calculate $E[X^2]$. \emph{Hint:} recall that $b^3-a^3$ can be factorized as $(b-a)(b^2 + a^2 + ab)$.
			\begin{solution}
				\begin{eqnarray*}
					E[X^2]& =& \int_{-\infty}^\infty x^2 f(x) \; dx = \int_a^b \frac{x^2}{b-a} = \left. \frac{x^3}{3(b-a)}\right|_a^b = \frac{b^3 - a^3}{3(b-a)}\\ \\
					&=& \frac{(b-a)(b^2 + a^2 + ab)}{3(b-a)} = \frac{b^2 + a^2 + ab}{3}
				\end{eqnarray*}
			\end{solution}
		\part Using the shortcut formula and parts (e) and (f),  show that $Var(X) = (b-a)^2/12$.
			\begin{solution}
				\begin{eqnarray*}
					Var(X) &=& E[X^2] - \left(E[X]\right)^2 = \frac{b^2 + a^2 + ab}{3} - \left(\frac{a+b}{2}\right)^2\\\\
					&=& \frac{b^2 + a^2 + ab}{3} -\frac{a^2 + 2ab + b^2}{4}\\\\
					&=& \frac{4b^2 + 4a^2 + 4ab - 3a^2 - 6ab - 3b^2}{12}\\\\
					&=& \frac{b^2 + a^2 - 2ab}{12} = \frac{(b-a)^2}{12}
				\end{eqnarray*}
			\end{solution}
	\end{parts}

	
\question Suppose that $X\sim N(0, 16)$ independent of $Y\sim N(2, 4)$. Recall that our convention is to express the normal distribution in terms of its mean and variance, i.e.\ $N(\mu, \sigma^2)$. Hence, $X$ has a mean of zero and variance of 16, while $Y$ has a mean of 2 and a variance of 4. In completing some parts of this question you will need to use the R function \texttt{pnorm} described in class. In this case, please write down the command you used as well as the numeric result.
	\begin{parts}
		\part Calculate $P( -8\leq X \leq 8)$.
			\begin{solution}
				\begin{eqnarray*}
				P( -8\leq X \leq 8) = P( -8/4\leq X/4 \leq 8/4) = P( -2\leq Z \leq 	2)\approx 0.95
				\end{eqnarray*}
where $Z$ is a standard normal random variable.
			\end{solution}
		\part Calculate $P(0 \leq Y \leq 4)$.
					\begin{solution}
			\begin{eqnarray*}
				P(0 \leq Y \leq 4) = P\left(\frac{0-2}{2} \leq \frac{Y-2}{2}\leq \frac{4-2}{2}\right) = P(-1 \leq Z \leq 1) \approx 0.68
			\end{eqnarray*}
			where $Z$ is a standard normal random variable.
			\end{solution}
		\part Calculate $P( -1\leq Y \leq 6)$.
					\begin{solution}
				\begin{eqnarray*}
					P( -1\leq Y \leq 6) &=& P\left( \frac{-1 -2}{2}\leq \frac{Y - 2}{2} \leq \frac{6-2}{2}\right)\\\\
					&=& P(-1.5 \leq Z \leq 2)\\
					&=& \Phi(2) - \Phi(-1.5)\\
					&=& \mbox{\texttt{pnorm(2) - pnorm(-1.5)}}\\
					&\approx& 0.91
				\end{eqnarray*}
				where $Z$ is a standard normal random variable.
			\end{solution}
		\part Calculate $P(X \geq 10)$.
					\begin{solution}
			\begin{eqnarray*}
				P(X \geq 10) &=& 1 - P(X \leq 10) = 1 - P(X /4\leq 10/4) = 1 - P(Z\leq 2.5) \\
				&=& 1 - \Phi(2.5) = 1 - \mbox{\texttt{pnorm(2.5)}}\\
				&\approx& 0.006
			\end{eqnarray*}
			\end{solution}
	\end{parts}	
	

\noindent \paragraph{Note:} \emph{In the following five questions  $X_1, X_2 \sim \mbox{ iid } N(\mu, \sigma^2)$, $Y = (X_1 - \mu)/\sigma$, $Z = (X_2 - \mu)/\sigma$.}


\question 
	\begin{parts}
		\part What is the distribution of $X_1 + X_2$?
			\begin{solution}
				$X_1 + X_2 \sim N(2\mu, 2\sigma^2)$
			\end{solution}
		\part Use R to calculate $P(X_1 + X_2 > 5)$ if $\mu = 5$ and $\sigma^2 = 50$.
			\begin{solution}
			In this case, $X_1 + X_2 \sim N(10, 100)$, hence
			\begin{eqnarray*}
				P(X_1 + X_2 > 5) &=& 1 - P(X_1 + X_2 \leq 5) \\
				&=& 1 - P\left(\frac{X_1 + X_2 - 10}{10} \leq \frac{5-10}{10}\right)\\
				&=& 1 - \mbox{\texttt{pnorm(-0.5)}}\\
				&\approx& 0.6914625
			\end{eqnarray*}
			Alternatively, we could use  \texttt{1- pnorm(5, mean = 10, sd = 10)}, which gives the same result.
			\end{solution}
		\part Use R to calculate the 10th percentile of the distribution of $X_1 + X_2$.
			\begin{solution}
				\texttt{qnorm(p = 0.1, mean = 10, sd = 10)} gives -2.815516.
			\end{solution}
	\end{parts}
\question 
	\begin{parts}
		\part What is the distribution of $Y^2$?
			\begin{solution}
				As the sum of squares of one standard normal RV, $Y^2 \sim \chi^2(1)$.
			\end{solution}
		\part Use R to calculate $P(Y^2 \geq 1)$.
			\begin{solution}
				\begin{eqnarray*}
					P(Y^2 \geq 1) = 1 - P(Y^2 \leq 1) = 1 -  \mbox{\texttt{pchisq(1, df = 1)}} \approx 0.3173105
				\end{eqnarray*}
			\end{solution}
	\end{parts}
	
\question
	\begin{parts}
		\part What is the distribution of $Y^2 + Z^2$?
			\begin{solution}
			Since this is the sum of squares of two independent standard normal random variables, $Y^2 + Z^2 \sim \chi^2(2)$.
			\end{solution}
		\part Use R to calculate the 95th percentile of the distribution of $Y^2 + Z^2$.
			\begin{solution}
				\texttt{qchisq(p = 0.95, df = 2)} gives 5.991465
			\end{solution}
	\end{parts}
\question 
	\begin{parts}
	\part What is the distribution of $Z/\sqrt{Y^2}$?
		\begin{solution}
			Since it is the ratio of a standard normal to the square root of an independent $\chi^2$ random variable divided by its degrees of freedom (in this case one), $Z/\sqrt{Y^2}\sim t(1)$.
		\end{solution}
	\part What value of $c$ satisfies $P(-c\ \leq Z/\sqrt{Y^2} \leq c )=0.95$?
		\begin{solution}
			By the symmetry of the $t$-distribution, it suffices to find the $97.5$th percentile (this allocates 2.5\% probability to the upper and lower tails).  The command \texttt{qt(p = 0.975, df = 1)} gives 12.7062, so $c \approx 12.7$. Alternatively, we could have calculated the 2.5th percentile: \texttt{qt(p = 0.025, df = 1)} gives -12.7062.
		\end{solution}
	\part How does the interval in part (b) compare to the corresponding interval for $Z$?
		\begin{solution}
		Since $Z$ is a standard normal RV, $P(-2\leq Z \leq 2) \approx 0.95$. We see that the interval for a $t(1)$ RV is \emph{much wider} than the corresponding interval for a standard normal. In other words, extreme outcomes are much more likely under the $t(1)$ distribution.
		\end{solution}
	\end{parts}
	
\question 
	\begin{parts}
		\part What is the distribution of $Y^2/Z^2$?
			\begin{solution}
				This is the ratio of two independent $\chi^2$ random variables, each divided by its degrees of freedom (in this case, one). Hence $Y^2/Z^2\sim F(1,1)$.
			\end{solution}
		\part Use R to calculate the 95th percentile of the distribution of $Y^2/Z^2$.
			\begin{solution}
				\texttt{qf(p = 0.95, df1 = 1, df2 = 1)} gives 161.4476
			\end{solution}
	\end{parts}

\end{questions}


\end{document}
