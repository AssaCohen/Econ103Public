\documentclass[addpoints,12pt]{exam}
\usepackage{amsmath, amssymb}
\linespread{1.1}
\usepackage{hyperref}
%\boxedpoints
%\pointsinmargin


\newcommand{\p}{\mathbb{P}}
\newcommand{\expect}{\mathbb{E}}
\newcommand{\var}{\mathbb{V}}
\newcommand{\cov}{Cov}
\newcommand{\cprob}{\rightarrow_{p}}
\newcommand{\cas}{\rightarrow_{as}}
\newcommand{\clp}{\rightarrow_{L^p}}
\newcommand{\clone}{\rightarrow_{L^1}}
\newcommand{\cltwo}{\rightarrow_{L^2}}
\newcommand{\cd}{\rightarrow_{d}}
\newcommand{\cv}{\Rightarrow_{v}}
\newcommand{\dec}{\downarrow}
\newcommand{\inc}{\uparrow}
\newcommand{\plim}{\hbox{plim}_{n\rightarrow \infty}}
\newcommand{\limn}{\lim_{n \rightarrow \infty}}
\newcommand{\fil}{(\mathcal{F}_n)_{n=0}^{\infty}}
\newcommand{\xn}{(X_n)_{n=0}^{\infty}}
\newcommand{\hn}{(H_n)_{n=0}^{\infty}}
\DeclareMathOperator*{\argmax}{arg\,max}
\DeclareMathOperator*{\argmin}{arg\,min}
\newcommand{\norm}[1]{\left|\left|#1\right|\right|}
\newcommand{\inprod}[1]{\left\langle#1\right\rangle}
\newcommand{\slfrac}[2]{\left.#1\right/#2}


%\printanswers
%\noprintanswers

\title{Problem Set \#2}
\author{Econ 103}
\date{}
\begin{document}
\maketitle

\section*{Part I -- Problems from the Textbook}
Chapter 2: 1, 3, 5, 7, 13, 15, 17 [in part (b) skip MAD and MSD], 21 23, 27, 35. 

% \section*{Part II -- R Tutorial}
% Complete ``R Tutorial \#2'' available at the following address:\\ \url{http://www.ditraglia.com/econ103/Rtutorial2.html} 

\section*{Part II -- Additional Problems}
\begin{questions}

\question For each variable indicate whether it is nominal, ordinal, or numeric.
	\begin{parts}
		\part Grade of meat: prime, choice, good.
			\begin{solution}
				ordinal
			\end{solution}
		\part Type of house: split-level, ranch, colonial, other.
			\begin{solution}
				categorical
			\end{solution}
		\part Income
			\begin{solution}
			 numeric
			\end{solution}
	\end{parts}
	

	
\question A drive-time radio show frequently holds call-in polls during the evening rush hour. Explain in no more than two sentences why such polls are likely to be biased.
	\begin{solution}
		People who are listening to the radio during rush hour are disproportionately likely to be commuters driving home from work. People who are employed and drive to work are not representative of the population at large.  
	\end{solution}

	
\question Which of these studies are based on experimental data? Which are based on observational data?
	\begin{parts}
		\part A biologist examines fish in a river to determine the proportion that show signs of disease due to pollutants poured into the river upstream.
		\begin{solution}
		Observational
		\end{solution}
		\part In a pilot phase of a fund-raising campaign, a university randomly contacts half of a group of alumni by phone and the other half by a personal letter to determine which method results in higher contributions.
				\begin{solution}
				Experimental
		\end{solution}
		\part To analyze possible problems from the by-products of gas combustion, people with with respiratory problems are matched by age and sex to people without respiratory problems and then asked whether or not they cook on a gas stove.
				\begin{solution}
				Observational
		\end{solution}
		\part An industrial pump manufacturer monitors warranty claims and surveys customers to assess the failure rate of its pumps.
				\begin{solution}
				Observational
		\end{solution}
	\end{parts}



\question An emergency room institutes a new screening procedure to identify people suffering from life-threatening heart problems so that treatment can be initiated quickly. The procedure is credited with saving lives because in the first year after its initiation, there is a lower death rate due to heart failure compared to the previous year among patients seen in the emergency room. Do you agree? Explain.
\begin{solution}
	No. There could be many other reasons why death rates decreased, including improved medical technology in other areas. It could also be that the patients who came into the ER in the second year happened to be less sick, on average. In other words, there are many possible confounders.
\end{solution}



\question Suppose that $x_i$ is measured in centimeters and $y_i$ is measured in feet. What are the units of the following quantities? 
	\begin{parts}
		\part Interquartile Range of $x$
			\begin{solution}
	 centimeters
	\end{solution}
		\part Covariance between $x$ and $y$			
		\begin{solution}
	 centimeters $\times$ feet
	\end{solution}
		\part Correlation between $x$ and $y$
		\begin{solution}
		unitless
		\end{solution}
		\part Skewness of $x$
		\begin{solution}
		unitless
		\end{solution}
		\part Variance of $y$
		\begin{solution}
		feet$^2$	
		\end{solution}
	\end{parts}


	
\question The \emph{mean deviation} is a measure of dispersion that we did not cover in class. It is defined as follows:
	$$MD = \frac{1}{n}\sum_{i=1}^n |x_i - \bar{x}|$$
	\begin{parts}
		\part Explain why this formula averages the absolute value of deviations from the mean rather than the deviations themselves.
		\begin{solution}
		As we showed in class, the average deviation from the sample mean is zero regardless of the dataset. Taking the absolute value is similar to squaring the deviations: it makes sure that the positive ones don't cancel out the negative ones.
		\end{solution}
		\part Which would you expect to be more sensitive to outliers: the mean deviation or the variance? Explain.
		\begin{solution}
		The variance is calculated from squared deviations. When $x$ is far from zero, $x^2$ is much larger than $|x|$ so large deviations ``count more'' when calculating the variance. Thus, the variance will be more sensitive to outliers. 
		\end{solution}
	\end{parts}
	

\question Consider a dataset $x_1, \hdots, x_n$. Suppose I multiply each observation by a constant $d$ and then add another constant $c$, so that $x_i$ is replaced by $c + dx_i$.
	\begin{parts}
		\part How does this change the sample mean? Prove your answer.
			\begin{solution}
				\begin{eqnarray*}
					\frac{1}{n} \sum_{i=1}^n (c + dx_i)&=&\frac{1}{n} \sum_{i=1}^n c + d \left(\frac{1}{n} \sum_{i=1}^n x_i\right) = c + d\bar{x}
				\end{eqnarray*}
			\end{solution}
		\part How does this change the sample variance? Prove your answer.
		\begin{solution}
			$$\frac{1}{n-1} \sum_{i=1}^n [(c + dx_i) - (c + d\bar{x})]^2 = \frac{1}{n-1} \sum_{i=1}^n [d(x_i - \bar{x})]^2 = d^2 s_x^2$$
		\end{solution}
		\part How does this change the sample standard deviation? Prove your answer.
			\begin{solution}
			The new standard deviation is $|d| s_x$, the positive square root of the variance.
			\end{solution}
		\part How does this change the sample z-scores? Prove your answer.
			\begin{solution}
			They are unchanged:
				$$\frac{(c + d x_i) - (c + d\bar{x})}{d s_x} = \frac{d (x_i - \bar{x})}{d s_x} = \frac{x_i - \bar{x}}{s_x}$$
			\end{solution}
	\end{parts}

	
\end{questions}










































\end{document}