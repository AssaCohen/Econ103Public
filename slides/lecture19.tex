\section{Test for the mean of a normal population (variance known)}
%%%%%%%%%%%%%%%%%%%%%%%%%%%%%%%%%%%%%%%%
\begin{frame}
  \frametitle{A Simple Example}
  
  Suppose $X_1, \dots, X_{100} \sim \mbox{ iid N}(\mu, \sigma^2 = 9)$ and we want to test
  \[
    \begin{array}{c}
      H_0\colon \mu = 2\\
      H_1\colon \mu \neq 2\\
    \end{array}
  \]

  \pause

  \begin{enumerate}
    \item[Step 1] -- Specify Null Hypothesis $H_0$ and alternative Hypothesis $H_1$ $\textcolor{blue}{\checkmark}$\pause
    \item[Step 2] -- \alert{Choose Test Statistic $T_n$}
  \end{enumerate}

  \pause

  If $\bar{X}$ is far from 2 then $\mu=2$ is implausible. Why?

\end{frame}
%%%%%%%%%%%%%%%%%%%%%%%%%%%%%%%%%%%%%%%%
\begin{frame}
  \frametitle{If $\bar{X}_n$ is far from 2, then $\mu = 2$ is implausible}

  Since $X_1, \dots, X_{100} \sim \mbox{ iid N}(\mu, 9)$, \alert{if $\mu = 2$ then $\bar{X} \sim N(2, 0.09)$}
  \begin{eqnarray*}
     P(a \leq \bar{X} \leq b) &=& \pause P\left(\frac{a - 2}{3/10} \leq \frac{\bar{X}-2}{3/10} \leq \frac{b - 2}{3/10} \right)\\ \pause
     &=& P\left( \frac{a-2}{0.3} \leq Z \leq \frac{b-2}{0.3} \right)
  \end{eqnarray*}
  where $Z \sim N(0,1)$ so we see that if $H_0\colon \mu = 2$ is true then \pause
  \begin{eqnarray*}
    P(1.7 \leq \bar{X} \leq 2.3) &=& P(-1 \leq Z \leq 1) \approx 0.68\\ \pause
    P(1.4 \leq \bar{X} \leq 2.6) &=& P(-2 \leq Z \leq 2) \approx 0.95 \\ \pause
    P(1.1 \leq \bar{X} \leq 2.9) &=& P(-3 \leq Z \leq 3) > 0.99 
  \end{eqnarray*}

\end{frame}
%%%%%%%%%%%%%%%%%%%%%%%%%%%%%%%%%%%%%%%%
\begin{frame}
  \frametitle{Step 2 -- Choose Test Statistic $T_n$}
  \begin{itemize}
    \item Reject $H_0\colon \mu = 2$ if the sample mean is far from $2$. \pause
    \item $\Rightarrow T_n$ should depend on the \alert{distance} from $\bar{X}$ to 2, i.e.\ $|\bar{X} - 2|$.\pause
    \item We can make our subsequent calculations much easier if we choose a \alert{scale for $T_n$ that is convenient under $H_0$\dots}
  \end{itemize}
  \begin{eqnarray*}
    \mu=2 \Rightarrow \quad \bar{X} - 2 &\sim& N(0, 0.09) \\ \\ \pause
    \frac{\bar{X} - 2}{0.3} &\sim& N(0,1)
  \end{eqnarray*}

  \alert{So we will set $\displaystyle T_n = \left|\frac{\bar{X} - 2}{0.3}\right|$}

\end{frame}
%%%%%%%%%%%%%%%%%%%%%%%%%%%%%%%%%%%%%%%%
\begin{frame}
  \frametitle{A Simple Example: $X_1, \dots, X_{100}\sim \mbox{iid N}(\mu, \sigma^2 = 9)$}
  \begin{enumerate}
    \item[Step 1] -- $H_0\colon \mu = 2, \; H_1\colon \mu \neq 2$ $\textcolor{blue}{\checkmark}$
    \item[Step 2] -- $T_n = \displaystyle \left|\frac{\bar{X} - 2}{0.3} \right|$ $\textcolor{blue}{\checkmark}$
    \item[Step 3] -- If $\mu = 2$ then $\displaystyle \left(\frac{\bar{X} - 2}{0.3}\right) \sim N(0,1)$ $\textcolor{blue}{\checkmark}$
    \item[Step 4] -- \alert{Choose Critical Value $c$}
      \begin{enumerate}[(i)]
        \item Specify significance level $\alpha$.
        \item Choose $c$ so that $P(T_n > c)=\alpha$ under $H_0\colon \mu = 2$.
      \end{enumerate}
  \end{enumerate}
\end{frame}

%%%%%%%%%%%%%%%%%%%%%%%%%%%%%%%%%%%%%%%%
\begin{frame}
  \frametitle{Choose $c$ so that $P(T_n >c) = \alpha$ under $H_0$}
  $T_n = \displaystyle \left|\frac{\bar{X}-2}{0.3}\right|$ and $\mu=2 \implies \displaystyle \frac{\bar{X}-2}{0.3} \sim N(0,1)$ \pause

  \begin{eqnarray*}
    P\left( \left|\frac{\bar{X}-2}{0.3}\right| > c \right) &=& \alpha\\ \pause
    1 - P\left( \left|\frac{\bar{X}-2}{0.3}\right| \leq c \right) &=& \alpha\\ \pause
    P\left( \left|\frac{\bar{X}-2}{0.3}\right| \leq c \right) &=& 1 - \alpha\\ \pause
    P\left(-c \leq \frac{\bar{X}-2}{0.3} \leq c \right) &=& 1 - \alpha
  \end{eqnarray*}

  \alert{Hence: $c = \texttt{qnorm}(1 - \alpha/2)$ which should look familiar!}
\end{frame}
%%%%%%%%%%%%%%%%%%%%%%%%%%%%%%%%%%%%%%%%
\begin{frame}
  \frametitle{A Simple Example: $X_1, \dots, X_{100}\sim \mbox{iid N}(\mu, \sigma^2 = 9)$}
  \begin{enumerate}
    \item[Step 1] -- $H_0\colon \mu = 2, \; H_1\colon \mu \neq 2$ $\textcolor{blue}{\checkmark}$
    \item[Step 2] -- $T_n = \displaystyle \left|\frac{\bar{X} - 2}{0.3} \right|$ $\textcolor{blue}{\checkmark}$
    \item[Step 3] -- If $\mu = 2$ then $\displaystyle \left(\frac{\bar{X} - 2}{0.3}\right) \sim N(0,1)$ $\textcolor{blue}{\checkmark}$
    \item[Step 4] -- $c = \texttt{qnorm}(1 - \alpha /2)$ $ \textcolor{blue}{\checkmark}$
    \item[Step 5] -- \alert{Look at the data: if $T_n >c$, reject $H_0$}\pause
      \begin{itemize}
        \item Suppose I choose $\alpha = 0.05$. Then $c \approx 2$.\pause
        \item I observe a sample of 100 observations. Suppose $\bar{x} = 1.34$
          \[T_n = \displaystyle\left|\frac{\bar{x} - 2}{0.3}\right| =\left|\frac{1.34 - 2}{0.3}\right| = 2.2  \]\pause
          \vspace{-2em}
        \item Since $T_n > c$, I reject $H_0\colon \mu=2$.
      \end{itemize}
  \end{enumerate}
\end{frame}

%%%%%%%%%%%%%%%%%%%%%%%%%%%%%%%%%%%%%%%%
\begin{frame}
  \frametitle{Reporting the Results of a Test}

  \begin{block}{Our Example: $X_1, \dots, X_{100} \sim \mbox{iid N}(\mu, 1)$}
    
    \begin{itemize}
      \item $H_0\colon \mu = 2$ vs.\ $H_1\colon \mu \neq 2$
      \item $T_n = |(\bar{X}_n - 2)/0.3|$
      \item $\alpha = 0.05 \implies c \approx 2$
    \end{itemize}
  \end{block}

  \pause

  \begin{block}{Suppose $\bar{x}=1.34$}
    Then $T_n = 2.2$. Since this is greater than $c$ for $\alpha = 0.05$, we \alert{reject $H_0\colon \mu=2$ at the 5\% significance level.}
  \end{block}

  \pause

  \begin{block}{Suppose instead that $\bar{x}=1.82$}
   Then $T_n = 0.6$.
   Since this is less than $c$ for $\alpha = 0.05$, we \alert{fail to reject $H_0\colon \mu = 2$ at the 5\% signifcance level.}
  \end{block}


\end{frame}
%%%%%%%%%%%%%%%%%%%%%%%%%%%%%%%%%%%%%%%%
\begin{frame}
  \frametitle{General Version of Preceding Example}

 $X_1, \dots, X_n \sim \mbox{iid N}(\mu, \sigma^2)$ with $\sigma^2$ known and we want to test:
  \[
    \begin{array}{c}
      H_0\colon \mu = \mu_0\\
      H_1\colon \mu \neq \mu_0\\
    \end{array}
  \]
  where $\mu_0$ is some specified value for the population mean.

  \pause
  
  \begin{itemize}
    \item $|\bar{X}_n - \mu_0|$ tells how far sample mean is from $\mu_0$. \pause
    \item Reject $H_0\colon \mu=\mu_0$ if sample mean is far from $\mu_0$. \pause
    \item Under $H_0\colon \mu = \mu_0$, $\displaystyle\frac{\bar{X}_n - \mu_0}{\sigma/\sqrt{n}} \sim N(0,1)$. \pause
    \item Test statistic $T_n = \displaystyle\left|\frac{\bar{X}_n - \mu_0}{\sigma/\sqrt{n}}\right|$ \pause
    \item Reject $H_0\colon \mu = \mu_0$ if $T_n > \texttt{qnorm}(1 - \alpha/2)$
  \end{itemize}

  
  
\end{frame}
%%%%%%%%%%%%%%%%%%%%%%%%%%%%%%%%%%%%%%%%
\section{Relationship Between Confidence Intervals and Hypothesis Tests}
%%%%%%%%%%%%%%%%%%%%%%%%%%%%%%%%%%%%%%%%
\begin{frame}
  \frametitle{What is this test telling us to do?}
  Return to specific example where $H_0\colon \mu = 2$ vs.\ $H_1\colon \mu \neq 2$ and $X_1, \dots, X_{100} \sim \mbox{iid N}(\mu, 1)$ with $\alpha = 0.05$:
  \begin{eqnarray*}
    \mbox{Reject } H_0 &\mbox{ if }& \left|\frac{\bar{X}_n - 2}{0.3}\right| > 2\\\pause
     \mbox{Reject } H_0 &\mbox{ if }&|\bar{X}_n - 2| > 0.6 \\ \pause
     \alert{\mbox{Reject } H_0} &\alert{\mbox{ if }}& \alert{(\bar{X_n} < 1.4) \mbox{ or } (\bar{X}_n > 2.6)} 
  \end{eqnarray*}

  Reject $H_0\colon \mu = 2$ if $\bar{X_n}$ is far from $2$. 
  How far?
  Depends on choice of $\alpha$ along with sample size and population variance.
\end{frame}
%%%%%%%%%%%%%%%%%%%%%%%%%%%%%%%%%%%%%%%%
\begin{frame}
  \frametitle{This looks suspiciously similar to a confidence interval\dots}

  \small 
  \[
    \boxed{X_1, \dots, X_n \sim \mbox{iid N}(\mu, \sigma^2) \mbox{ where }\sigma^2 \mbox{ is known}}
  \]
  \[
    \boxed{T_n = \displaystyle\left|\frac{\bar{X}_n - \mu_0}{\sigma/\sqrt{n}}\right|, \; c = \texttt{qnorm}(1 - \alpha/2), \; \mbox{Reject } H_0\colon \mu = \mu_0 \mbox{ if } T_n > c}
  \]

  \vspace{1em}
  Another way of saying this is don't reject $H_0$ if:
  \begin{eqnarray*}
    \left(T_n \leq c\right) \pause &\iff&
    \left(\left|\frac{\bar{X}_n - \mu_0}{\sigma/\sqrt{n}}\right| \leq c \right) \pause
    \iff \left(-c \leq \frac{\bar{X}_n - \mu_0}{\sigma/\sqrt{n}}\leq c\right)\\ \pause
    &\iff& \left(\bar{X}_n - c \times \frac{\sigma}{\sqrt{n}} \leq \mu_0 \leq \bar{X}_n + c\times\frac{\sigma}{\sqrt{n}}  \right)
  \end{eqnarray*}

  \pause

  \alert{In other words, don't reject $H_0\colon \mu = \mu_0$ at significance level $\alpha$ if $\mu_0$ lies inside the $100 \times (1 - \alpha)\%$ confidence interval for $\mu$.}

\end{frame}
%%%%%%%%%%%%%%%%%%%%%%%%%%%%%%%%%%%%%%%%
\begin{frame}
  \frametitle{CIs and Hypothesis Tests are Intimately Related}

  \begin{block}{Our Simple Example}
    $X_1, \dots, X_{100} \sim \mbox{iid N}(\mu, \sigma^2 = 9)$ and observe $\bar{x} = 1.34$ 
  \end{block}

  \pause

  \begin{block}{Test $H_0\colon \mu = 2$ vs.\ $H_1\colon \mu \neq 2$ with $\alpha = 0.05$}
    $T_n = 2.2$, $c = \texttt{qnorm}(1 - 0.05/2) \approx 2$. Since $T_n>c$ we reject.
    
  \end{block}

  \pause

  \begin{block}{95\% Confidence Interval for $\mu$} 
    $1.34 \pm 2 \times 3 / 10$ i.e.\ $1.34 \pm 0.6$ or equivalently $(0.74, 1.94)$
  \end{block}

  \pause

  \begin{block}{Another way to carry out the test\dots}
   Since 2 lies outside the 95\% confidence interval for $\mu$, if our significance level is $\alpha = 0.05$ we reject $H_0\colon \mu = 2$.
  \end{block}

\end{frame}
%%%%%%%%%%%%%%%%%%%%%%%%%%%%%%%%%%%%%%%%
\section{P-values}
%%%%%%%%%%%%%%%%%%%%%%%%%%%%%%%%%%%%%%%%
\begin{frame}
  \frametitle{$X_1, \dots X_{100} \sim \mbox{iid N}(\mu_X, 1)$ and $Y_1, \dots, Y_{100}\sim \mbox{iid N}(\mu_Y, 1)$}
  Two researchers: $H_0\colon \mu = 2$ vs.\ $H_1\colon \mu \neq 2$ with $\alpha = 0.05$

\begin{columns}
  \column{0.45\textwidth}
  \begin{block}{Researcher 1}
    \begin{itemize}
      \item $\bar{x} = 1.34$
      \item $T_n = 2.2 > 2$
      \item Reject $H_0\colon \mu_X = 2$ 
    \end{itemize}
  \end{block}

  \column{0.45\textwidth}
  \begin{block}{Researcher 2}
    \begin{itemize}
      \item $\bar{y} = 11.3$
      \item $T_n = 31 > 2$
      \item Reject $H_0\colon \mu_Y = 2$
    \end{itemize}
  \end{block}
\end{columns}

\vspace{1em}
\alert{Both researchers would report ``reject $H_0$ at the 5\% level'' but Researcher 2 found much stronger evidence against $H_0$\dots}

\end{frame}
%%%%%%%%%%%%%%%%%%%%%%%%%%%%%%%%%%%%%%%%
\begin{frame}
  \frametitle{What if we had chosen a different significance level $\alpha$?}

  \vspace{-2em}

  \[\boxed{T_n = 2.2, \quad c = \texttt{qnorm}(1 -\alpha/2), \quad \mbox{Reject } H_0\colon \mu = 2 \mbox{ if } T_n > c}\]

  \vspace{-2em}

  \pause

  \begin{eqnarray*}
    \alpha = 0.32 &\Rightarrow& c = \texttt{qnorm}(1 - 0.32/2) \approx 0.99 \quad \mbox{\textcolor{blue}{Reject}}\\ \pause
    \alpha = 0.10 &\Rightarrow& c = \texttt{qnorm}(1 - 0.10/2) \approx 1.64\quad \mbox{\textcolor{blue}{Reject}}\\  \pause
    \alpha = 0.05 &\Rightarrow& c = \texttt{qnorm}(1 - 0.05/2) \approx 1.96 \quad \mbox{\textcolor{blue}{Reject}}\\ \pause
    \alpha = 0.04 &\Rightarrow& c = \texttt{qnorm}(1 - 0.04/2) \approx 2.05\quad \mbox{\textcolor{blue}{Reject}}\\  \pause
    \alpha = 0.03 &\Rightarrow& c = \texttt{qnorm}(1 - 0.03/2) \approx 2.17\quad \mbox{\textcolor{blue}{Reject}}\\  \pause
    \alpha = 0.02 &\Rightarrow& c = \texttt{qnorm}(1 - 0.02/2) \approx 2.33 \quad \mbox{\textcolor{red}{Fail to Reject}}\\
    \alpha = 0.01 &\Rightarrow& c = \texttt{qnorm}(1 - 0.01/2) \approx 2.58 \quad \mbox{\textcolor{red}{Fail to Reject}}
  \end{eqnarray*}

  
\end{frame}
%%%%%%%%%%%%%%%%%%%%%%%%%%%%%%%%%%%%%%%%
\begin{frame}
  \frametitle{Result of Test Depends on Choice of $\alpha$!}

  \begin{columns}
    \column{0.3\textwidth}
    \footnotesize
  \begin{eqnarray*}
    \alpha = 0.32 &\Rightarrow& \mbox{\textcolor{blue}{Reject}}\\ 
    \alpha = 0.10 &\Rightarrow& \mbox{\textcolor{blue}{Reject}}\\
    \alpha = 0.05 &\Rightarrow& \mbox{\textcolor{blue}{Reject}}\\ 
    \alpha = 0.04 &\Rightarrow& \mbox{\textcolor{blue}{Reject}}\\
    \alpha = 0.03 &\Rightarrow& \mbox{\textcolor{blue}{Reject}}\\
    \alpha = 0.02 &\Rightarrow& \mbox{\textcolor{red}{Fail to Reject}}\\
    \alpha = 0.01 &\Rightarrow& \mbox{\textcolor{red}{Fail to Reject}}
  \end{eqnarray*}
    
    \column{0.6\textwidth}

    \begin{itemize}
      \item If you reject $H_0$ at a given choice of $\alpha$, you would also have rejected at any \alert{larger} choice of $\alpha$.\pause
      \item If you fail to reject $H_0$ at a given choice of $\alpha$, you would also have failed to reject at any \alert{smaller} choice of $\alpha$.
    \end{itemize}
    
  \end{columns}

  \vspace{1em}

  \pause

  \begin{block}{Question}
   If $\alpha$ is large enough we will reject; if $\alpha$ is small enough, we won't.
   Where is the \alert{dividing line} between reject and fail to reject?
  \end{block}

  
\end{frame}
%%%%%%%%%%%%%%%%%%%%%%%%%%%%%%%%%%%%%%%%
\begin{frame}
  \frametitle{P-Value: Dividing Line Between Reject and Fail to Reject}

  \vspace{-2em}

  \[\boxed{T_n = 2.2, \quad c = \texttt{qnorm}(1 -\alpha/2), \quad \mbox{Reject } H_0\colon \mu = 2 \mbox{ if } T_n > c}\]

  \begin{block}{Question}
    Given that we observed a test statistic of $2.2$, what choice of $\alpha$ would put us \alert{just at the cusp} of rejecting $H_0$? 
  \end{block}

  \pause

  \begin{alertblock}{Answer}
    Whichever $\alpha$ makes $c = 2.2$!
    At this $\alpha$ we just \alert{barely} fail to reject.
  \end{alertblock}


\end{frame}
%%%%%%%%%%%%%%%%%%%%%%%%%%%%%%%%%%%%%%%%
\begin{frame}
  \frametitle{Calculating the P-value}

  \begin{block}{Definition of a P-value}
    The significance level $\alpha$ such that the critical value $c$ for the test is \alert{exactly equal} to the observed value of the test statistic. 
  \end{block}

  \pause

  \begin{block}{Our Example}
    The observed value of the test statistic is $2.2$ and the critical value is $\texttt{qnorm}(1 - \alpha/2)$, so we need to solve:

    \vspace{-1em}

    \small
    \begin{eqnarray*}
      2.2 &=&  \texttt{qnorm}(1 - \alpha/2)\\ \pause
      \texttt{pnorm}(2.2) &=&  \texttt{pnorm}\left( \texttt{qnorm}\left( 1 - \alpha/2 \right) \right) \\ \pause
      \texttt{pnorm}(2.2) &=&  1 - \alpha/2\\ \pause
      \alpha &=& 2 \times \left[ 1 - \texttt{pnorm}(2.2) \right] \approx 0.028
    \end{eqnarray*}
  \end{block}



\end{frame}
%%%%%%%%%%%%%%%%%%%%%%%%%%%%%%%%%%%%%%%%
\begin{frame}
  \frametitle{How to use a p-value?}
  How to report results based on p-value. 
  How p-value can be used to carry out a test with given $\alpha$
\end{frame}
%%%%%%%%%%%%%%%%%%%%%%%%%%%%%%%%%%%%%%%%
\section{One-Sided Tests}
%%%%%%%%%%%%%%%%%%%%%%%%%%%%%%%%%%%%%%%%
\begin{frame}
  \frametitle{One-sided Tests} 
  These are stranger since the analogy with a confidence interval breaks down.
  Explain that the default should be a two-sided test.
  Then explain why and when you might want to do a one-sided test.
  It's about \emph{power} which we'll cover in more detail in the next lecture. 
  But the basic idea is that you're more likely to reject if the null is false.
  Show an example.
  Then explain about a one-sided p-value.
\end{frame}
%%%%%%%%%%%%%%%%%%%%%%%%%%%%%%%%%%%%%%%%
\begin{frame}
  \frametitle{Roadmap}

  \begin{block}{Next Time}
    More examples of hypothesis testing, using relationship with confidence intervals to help us.
  \end{block}


  \pause

  \begin{block}{Building Intuition}
   Now that you know a simple example of a hypothesis test and its relationship to a CI, think about the following:
   \begin{itemize}
     \item If we reject $H_0$ does that mean that $H_0$ is false?
     \item How does testing relate to random sampling?
      \item How does critical value of a test relate to width of a CI?
   \end{itemize}
  \end{block}

\end{frame}
%%%%%%%%%%%%%%%%%%%%%%%%%%%%%%%%%%%%%%%%
