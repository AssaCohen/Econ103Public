\documentclass[11pt, letterpaper]{article}
\usepackage{geometry}
\geometry{margin=1in} 
\usepackage{setspace}
\linespread{1}
\usepackage{hyperref}
\usepackage{enumerate}
\usepackage{fancybox}
\usepackage{amsmath, amssymb}




%%% HEADERS & FOOTERS
\usepackage{fancyhdr} 
\pagestyle{fancy} % options: empty , plain , fancy
\renewcommand{\headrulewidth}{0.4pt} % customise the layout...
\rhead{\footnotesize DiTraglia -- Spring 2015}
\lhead{\footnotesize Econ 103 Syllabus}
\renewcommand\footrulewidth{0pt}


\begin{document}


\thispagestyle{plain}

\begin{center}
\Large
\sc
Statistics for Economists\\
\large
Economics 103\\
\large
Spring 2015
\end{center}


\normalsize
\bigskip
\noindent \textbf{Course Instructor:} Francis DiTraglia 

\medskip


\noindent \textbf{Recitation Instructors:}
  Rodrigo Azuero Melo, 
  Rossa O'Keeffe-O'Donovan,
  Yiwen Wu
\medskip


\noindent \textbf{Office Hours:} \emph{For times and locations, see the semester calendar on the course website.}


\medskip
 
\noindent \textbf{Course Website:} \url{http://ditraglia.com/Econ103Public} At this url you can view the semester calendar and download all lecture slides, problem sets, etc.
You can view your grades and log-on to the course discussion board, Piazza, at \url{https://canvas.upenn.edu}.

\medskip

\noindent \textbf{Course Email:} To ensure a timely response, please direct all email related to this course to \href{mailto:econ103penn@gmail.com}{\texttt{econ103penn@gmail.com}} rather than your instructor or RI's personal account.
Please note that the course email account is reserved for \emph{personal issues only}. 
Please direct all questions concerning course material or logistics to the course discussion board, Piazza.

\medskip



\noindent \textbf{Course Description:} 
This course will teach you how to learn from data and understand uncertainty using the ideas of probability theory and statistics. 
After completing this course you will be able to carry out simple statistical analyses of your own using the computer package R.


\medskip

\noindent \textbf{Prerequisites:} 
The prerequisite for this course is multivariate calculus (Math 104 followed by 114 or 115). 
To do well in this course you will need to be comfortable with algebra, manipulating sums, differentiation and partial differentiation, solving unconstrained optimization problems, and integration. 
To help you gauge your level of mathematical preparation, we will administer a short math diagnostic quiz early in the semester.
This will not count towards your grade.




\medskip

\noindent \textbf{Required Text:} 
The required textbook for this course is \emph{Introductory Statistics for Business and Economics}, 4\textsuperscript{th} Edition by Thomas H.\ and Ronald J.\ Wonnacott (WW4). 
If you purchased this book from the bookstore, you got ripped off! I suggest that you return it and purchase a used copy on \href{http://tinyurl.com/ECON103-2013A}{Amazon}.
This book is ancient, so cheap used copies are plentiful.
The document ``Recommended Readings'' on the course website provides reading assignments to accompany each lecture.
While I strongly suggest that you complete the assigned readings, my lecture slides, which will be posted online after each lecture, are the final authority on course material. 
In particular, you are \emph{not} responsible for material in the textbook \emph{unless} it is also covered in lecture, but you \emph{are} responsible for material from lecture even if it is \emph{not} covered in the textbook.

\medskip


\noindent \textbf{Required Technology: } 
We will be using the ``clickers'' for experiments and class participation exercises during the semester.
Both the ResponseCard NXT and the ResponseCard QT are fully compatible with the exercises we will complete this semester.
Other ResponseCard models will only work for some of the exercises so to make sure you get ful l participation credit (see below) make sure you have the right model.
You can buy or rent a clicker from the Penn Bookstore.
As clicker participation will make up 5\% of your course grade it is important that you bring your clicker to each lecture. 
I fully understand, however, that things can go wrong: your clicker might stop working or you might forget to bring it after pulling an all-nighter.
For this reason you will \emph{automatically} be excused from clicker participation for four lectures: there is no need to inform us in advance or after the fact. There will, however, be no further exceptions. 
I will begin keeping track of clicker participation in our second lecture. For more details, see ``Participation'' below. 
Because clickers will determine a portion of your grade, their use is subject to the code of academic integrity, as explained below under ``Academic Integrity.'' 

\newpage


\noindent \textbf{Required Software:} 
We will use the statistical package R via a front-end called RStudio throughout the course. 
Both programs are free and open source. See the last page of this syllabus for instructions on how to configure your computer to run R and RStudio. Both programs are also available in the Undergraduate Data Analysis Lab (UDAL) in McNeil rooms 104 and 108--9. 
You will be taught to use R primarily through a series of tutorials that I will assign as homework. (See ``Homework'' below.)  
Additional R resources are listed on the last page of this syllabus.

\bigskip

\noindent \textbf{Recommended Texts:} 
I recommend two supplementary texts for students who feel they may need extra help with the course material. 
The first is the \emph{Student Workbook to accompany Introductory Statistics for Business and Economics 4\textsuperscript{th} Edition}. 
Used copies are available on \href{http://www.amazon.com/gp/offer-listing/0471508993/sr=/qid=/ref=olp_page_2?ie=UTF8&colid=&coliid=&condition=all&me=&qid=&shipPromoFilter=0&sort=sip&sr=&startIndex=10}{Amazon}. 
The workbook contains full solutions to all odd-numbered problems from the textbook, while the text itself provides answers but no explanations. 
The workbook also contains extra practice problems with solutions. The second recommended text is \emph{The R Student Companion} by Brian Dennis.
This text is intended for those students who are having trouble learning R and prefer a physical book to the free online resources listed at the end of this document. 

\bigskip

\noindent \textbf{Lecture Recordings: } 
Audio and screen captures of all lectures will be automatically recorded and posted on \href{http://upenn.instructure.com}{Canvas}. 
This is a great way to get caught up if you miss a lecture.

\bigskip

\noindent \textbf{Departmental Course Policies: } 
All Economics Department course policies are in force in Econ 103 even if not explicitly listed on this syllabus. 
See: \url{http://economics.sas.upenn.edu/undergraduate-program/course-information/guidelines/policies} for full details. 


\bigskip


\noindent \textbf{Academic Integrity: } 
All suspected violations of the code of academic integrity as set forth in the Pennbook will be reported to the Office of Student Conduct. 
Confirmed violations will result in failure for the course. 
Because it will be used to determine your class participation grade, operating a clicker on behalf of another student is cheating. 
If you are discovered using a clicker other than your own or have votes in a class that you did not attend, you will face the penalties described above. 
We will check identification cards at exams so please be sure to bring yours.

\bigskip

\noindent \textbf{Piazza:} 
We will be using an online discussion forum called Piazza for this course, which you can access directly from \href{http://upenn.instructure.com}{Canvas}. 
Piazza is where we will make course announcements and answer questions about course material and logistics.
By asking your question and getting an answer on Piazza, you create a positive externality: other students benefit from your questions and you benefit from theirs. 
The instructor and RIs will actively moderate Piazza both to answer questions and approve (or correct) answers written by your fellow-students.
As an incentive, I will award ``free points'' worth 5\% of your grade for making active use of the forum. 
See ``Participation'' below for more details.

\bigskip

\noindent \textbf{Attendance:} 
Regularly attending lectures is the only way to earn clicker participation points. 
As described above under ``Required Technology'' you will \emph{automatically} be excused from clicker participation for four lectures, but there will be no further exemptions of any kind. 
Similarly, regularly attending recitations is the only way to avoid a string of zeros on the quizzes (see ``Quizzes'' below). 
I will drop your two lowest quiz grades, including absences.

\bigskip

\newpage

\section*{Assignments and Grading}
	\begin{equation*}
	\begin{split}
		\mbox{Final Grade} = (&5\% \times \mbox{Clicker Participation}) + (5\% \times \mbox{Piazza Participation}) + (20\% \times \mbox{Quizzes})  \\ &
							 + (20\% \times \mbox{Midterm 1}) + (20\% \times \mbox{Midterm 2}) + (30\% \times \mbox{Final})
	\end{split}
	\end{equation*}
If necessary, I will curve final course grades (\emph{not} individual assignments) so that approximately 20-30\% fall in the A-range, 40-50\% fall in the B-range, and the bulk of the remaining 20-40\% fall in the C-range. 
I reserve grades below a C-minus for those rare cases in which a student fails to attain a minimum level of basic competence in statistics, an absolute rather than relative standard. 
If you are in danger of failing to meet this minimum standard, you will receive a course problem notice.
I will only curve the course in your favor, so the most stringent possible grade boundaries are: A-range = 90-100, B-range = 80-89, C-range = 70-79, D-range = 60-69.
(In this case, the top two points of each range would be a ``plus`` and the bottom two points a ``minus``.)

\medskip


\noindent \textbf{Clicker Participation:} 
Each lecture will feature activities in which you can earn participation credit by voting with your clicker. 
If you attend a given lecture and participate in the majority of the clicker activities, you will be counted as ``present.'' 
If you are ``present'' at at least 80\% of the semester's lectures, I will award you 100\% for clicker participation; otherwise I will deduct points proportionally. 
Hence, you are \emph{automatically} excused from two full weeks of lectures. 
This includes absences and forgotten or malfunctioning clickers. There will be no further exceptions.


\medskip

\noindent \textbf{Piazza Participation:} 
You will earn further participation credit based on the frequency and quality of your contributions on Piazza, including questions, answers, and follow-ups. 
If you participate actively, you will receive 100\%: these are essentially ``free points.''  
You \emph{must} contribute to earn points, but spamming the boards with unhelpful contributions will not gain you credit.  
Simply \emph{reading} posts on Piazza is not sufficient to earn participation points: you \emph{must} contribute.

\medskip



\noindent \textbf{Homework:} 
I have posted a number of problem sets and R Tutorials (with full solutions) on the course website.
Throughout the semester I will update the course calendar with ``due dates`` for these assignments.
Although homework will neither be collected nor graded, you should aim to complete each assignment by my suggested due date.
Unless you keep up with the course assignments, it will be impossible for you to do well on the exams and your course grade will suffer.
Be sure to use the solution keys responsibly: you gain nothing by simply reading them.


\medskip

\noindent \textbf{Quizzes:} 
Your RIs will administer a number of short quizzes in recitation over the course of the semester: dates appear on the semester calendar on the course website.
Each quiz will cover basic material from the most recent lectures since the last quiz or midterm. 
When calculating your quiz average, I will drop your two lowest scores and weight the remaining quizzes evenly. 
No makeup quizzes will be given so use your two ``free skips'' carefully.
Quizzes will not be returned and answers will not be posted but the RI's will go over each quiz in recitation.

\medskip

\noindent \textbf{Exams:} 
There will be two 70-minute in-class midterm exams and a 2-hour final exam during the exam period.
Dates, times and locations will appear on the semester calendar on the course website.
Each midterm is worth 20\% and the final is worth 30\% of your grade.
Neither midterm is comprehensive, but the final is: it will focuse on the final third of the course but include several questions on earlier material.
To give you a sense of the style and level of difficulty to expect, I have posted all of my past exams with full solutions on the course website.
Attendance at all exams is \emph{mandatory} and there will be no makeup midterms.
In exceptional circumstances, e.g.\ a death in the family or a serious documented illness, please contact the instructor in advance via the course email address.
The makeup final will take place at the beginning of next semester and is outside of the instructor's control: eligibility as well as the time and date are determined by the Economics Department. 
Exam regrade requests must be made in writing within a week of receiving your graded exam. 
As we re-grade the entire exam, your score could rise or fall. 
You may not discuss your answers with an RI or the instructor before submitting a regrade request. 
Exams will be photocopied before being returned and you may write in pencil or pen. 
Scientific calculators are permitted but graphing calculators are not. 
We will check ID cards at each exam.
            
\section*{Installing R and RStudio} First, download and install R from \url{http://cran.r-project.org/}. Second, download and install RStudio by visiting \url{http://rstudio.org/download/desktop} and clicking the link listed under ``Recommended for Your System.'' 

\bigskip

\section*{Additional R Resources} While not required, these references may be useful if you need some extra help learning R, or want to go beyond the material covered in the course.
\begin{itemize}
		       \item Contributed Documentation -- Comprehensive R Archive Network (CRAN) \\\url{http://cran.r-project.org/other-docs.html}
           	\begin{quote}
           		Comprehensive list of freely available reference material for R.
           	\end{quote}
\item R Twotorials -- Anthony Damico \\\url{http://www.twotorials.com/}
		\begin{quote}
		Ninety energetic, two-minute video tutorials on statistical programming with R. 
		\end{quote}
			\item Google Developers R Programming Video Lectures\\ \url{http://www.r-bloggers.com/google-developers-r-programming-video-lectures/}\begin{quote}R Programming video tutorials from beginning to advanced. \end{quote}
		 	\item Econometrics in R -- Grant Farnsworth\\\url{http://cran.r-project.org/doc/contrib/Farnsworth-EconometricsInR.pdf}
 		\begin{quote}
 		If you'd like to keep using R in Econ 104, this is what you should read.
 		\end{quote}
 			\item Resources to help you learn R -- UCLA Academic Technology Services \\\url{http://www.ats.ucla.edu/stat/R/}
		\begin{quote}
			A wealth of information about R, conveniently arranged in one place.
		\end{quote}
	           \item R in a Nutshell -- Adler\\ \url{http://proquestcombo.safaribooksonline.com/book/programming/r/9781449377502}         
           	\begin{quote}
           		Electronic version of the book of the same name published by O'Reilly (Accessible on the UPenn Network). Provides a comprehensive reference guide to R.
           	\end{quote}
		\item R-bloggers \\\url{http://www.r-bloggers.com}
		\begin{quote}
			A blog aggregator for R news and tutorials, with lots of applications.
		\end{quote}
\end{itemize}

\end{document}
