\documentclass[11pt, letterpaper]{article}
\usepackage{geometry}
\geometry{margin=1in} 
\usepackage{setspace}
\linespread{1}
\usepackage{hyperref}
\usepackage{enumerate}
\usepackage{fancybox}
\usepackage{amsmath, amssymb}




%%% HEADERS & FOOTERS
\usepackage{fancyhdr} 
\pagestyle{fancy} % options: empty , plain , fancy
\renewcommand{\headrulewidth}{0.4pt} % customise the layout...
\rhead{\footnotesize DiTraglia -- Spring 2016}
\lhead{\footnotesize Econ 103 Syllabus}
\renewcommand\footrulewidth{0pt}


\begin{document}


\thispagestyle{plain}

\begin{center}
\Large
\sc
Statistics for Economists\\
\large
Economics 103\\
\large
Spring 2017
\end{center}


\normalsize
\bigskip
\noindent \textbf{Course Instructor:} Francis DiTraglia 

\medskip


\noindent \textbf{Recitation Instructors:}
  Michael Chirico,
  Ethan (Mallick) Hossain,
  Rodrigo Azuero-Melo
\medskip


\noindent \textbf{Office Hours:} \emph{For times and locations, see the semester calendar on the \href{http://ditraglia.com/Econ103Public}{course website}.}


\medskip
 
\noindent \textbf{Course Website:} \url{http://ditraglia.com/Econ103Public} At this url you can view the semester calendar and download all lecture slides, problem sets, etc.
You can view your grades and log-on to the course discussion forum, Piazza, at \url{https://canvas.upenn.edu}.

\medskip

\noindent \textbf{Email Policy:}
Please direct all written communication concerning Econ 103 to the course discussion forum -- Piazza -- rather than to the instructors' personal email accounts.
For personal issues, use Piazza's private messaging feature to communicate directly with the course instructors. 

\medskip



\noindent \textbf{Course Description:} 
This course will teach you how to learn from data and understand uncertainty using the ideas of probability theory and statistics. 
After completing this course you will be able to carry out simple statistical analyses of your own using the computer package R.


\medskip

\noindent \textbf{Lecture Recordings:} 
Audio and screen captures of the lectures for this course will be recorded and posted on \href{http://upenn.instructure.com}{Canvas}. 
This is a great way to get caught up if you miss a lecture.
Be aware, however, that the lecture recording system is not 100\% reliable and the instructors have no direct control over it: in a typical semester one or two lectures will fail to be recorded due to technical problems.
If you encounter any problems with the recordings, please contact \texttt{mms-help@sas.upenn.edu} directly.
\medskip

\noindent \textbf{Prerequisites:} 
The prerequisites for this course are Math 104 followed by 114 or 115. 
To help you determine if this course is right for you we will administer a short math quiz early in the semester.
Although it will not count towards your grade, you must pass this quiz to be allowed to take the first midterm.
You may re-take the math quiz as many times as necessary until you pass.



\medskip

\noindent \textbf{Required Text:} 
The textbook for this course is \emph{Introductory Statistics for Business and Economics}, 4\textsuperscript{th} Edition by Thomas H.\ and Ronald J.\ Wonnacott (WW4). 
The price in the bookstore is outrageous; buy a used copy on \href{http://tinyurl.com/ECON103-2013A}{Amazon} instead. 
The document ``Recommended Readings'' on the \href{http://ditraglia.com/Econ103Public}{course website} provides reading assignments to accompany each lecture.
Any readings that do not come from the text will be posted on Piazza.
While I strongly encourage you to complete the reading assignments, my lecture slides are the final authority on course material.

\medskip


%\noindent \textbf{Required Technology: } 
%We will be using the ``clickers'' for experiments and class participation exercises during the semester.
%Both the ResponseCard NXT and the ResponseCard QT are fully compatible with the exercises we will complete this semester.
%Other ResponseCard models will only work for some of the exercises so to make sure you get full participation credit (see below) make sure you have the right model.
%You can buy or rent a clicker from the Penn Bookstore.
%As clicker participation will make up 5\% of your course grade it is important that you bring your clicker to each lecture. 
%I fully understand, however, that things can go wrong: your clicker might stop working or you might forget to bring it after pulling an all-nighter.
%For this reason you will \emph{automatically} be excused from clicker participation for four lectures: there is no need to inform us in advance or after the fact. There will, however, be no further exceptions. 
%I will begin keeping track of clicker participation in our second lecture. For more details, see ``Participation'' below. 
%Because clickers will determine a portion of your grade, their use is subject to the code of academic integrity, as explained below under ``Academic Integrity.'' 
%


\noindent \textbf{Required Software:} 
We will use the statistical package R via a front-end called RStudio throughout the course. 
Both R and RStudio are free and open source. Installation instructions appear on the last page of this syllabus.
RStudio is also available in the Undergraduate Data Analysis Lab (UDAL) in McNeil rooms 104 and 108--9. 
You will be taught to use R in lecture, recitations, and through as series of tutorials that I will assign as homework. (See ``Homework'' below.)  
Additional R resources are listed on the last page of this syllabus.

\medskip

\noindent \textbf{Recommended Texts:} 
I recommend two texts for students who need extra help with the course material. 
First is the \emph{Student Workbook to accompany Introductory Statistics for Business and Economics 4\textsuperscript{th} Edition}. 
Used copies are available on \href{http://www.amazon.com/gp/offer-listing/0471508993/sr=/qid=/ref=olp_page_2?ie=UTF8&colid=&coliid=&condition=all&me=&qid=&shipPromoFilter=0&sort=sip&sr=&startIndex=10}{Amazon}. 
The workbook contains fully worked out solutions to all odd-numbered problems from the textbook along with additional practice problems and solutions.
If you're having trouble with R and prefer a printed book to the free online resources listed below, I suggest consulting \emph{The R Student Companion} by Brian Dennis.


\newpage

\noindent \textbf{Departmental Course Policies: } 
All Economics Department course policies are in force in Econ 103 even if not explicitly listed on this syllabus. 
See: \url{http://economics.sas.upenn.edu/undergraduate-program/course-information/guidelines/policies} for full details. 


\bigskip


\noindent \textbf{Academic Integrity: } 
All suspected violations of the code of academic integrity as set forth in the Pennbook will be reported to the Office of Student Conduct. 
Confirmed violations will result in a failing grade for the course. 
We will check identification cards at exams so please to bring yours.

\medskip

\noindent \textbf{Piazza:} 
We will be using an online discussion forum called Piazza, accessible via \href{http://upenn.instructure.com}{Canvas}, for all written communication in this course.
We will use Piazza to make course announcements, answer questions about course material and respond to private messages from individual students regarding personal issues.
By asking your question and getting an answer on Piazza, you create a positive externality: other students benefit from your questions and you benefit from theirs.
You can even post anonymously if asking questions publicly makes you uncomfortable.
The instructor and RIs will actively moderate Piazza both to answer questions and approve (or correct) answers written by your fellow-students.
We will award extra credit for constructive questions, answers, and notes that you post on Piazza, even if you post anonymously.
See below under ``Extra Credit'' for details.
As mentioned above under ``Email Policy,'' all written communication for Econ 103 should be directed to Piazza, not to the instructors' personal email accounts.

\medskip

\noindent \textbf{Homework:} 
I will post homework and solutions on the \href{http://ditraglia.com/Econ103Public}{course website} throughout the semester. 
Although homework will neither be collected nor graded, it is \emph{crucial} that you keep up with the homework on a weekly basis if you hope to do well in the course.
Be sure to use the solution keys responsibly: you gain nothing by simply reading through the answers.



\section*{Assignments and Grading}

Grades for this course will be determined based on quizzes administered in recitation, two in-class midterms, and a comprehensive final examination:
	\begin{equation*}
	\begin{split}
		\mbox{Overall Score} = (30\% \times \mbox{Quizzes})  + (20\% \times \mbox{Midterm 1}) + (20\% \times \mbox{Midterm 2}) + (30\% \times \mbox{Final})
	\end{split}
	\end{equation*}
You can earn extra credit worth up to 5\% of your course grade: see ``Extra Credit'' for details.

\medskip 

\noindent \textbf{Course Curve:}
If necessary, I will curve overall course scores (\emph{not} individual assignments) so that approximately 30\% fall in the A-range, 40-50\% fall in the B-range, and the bulk of the remaining 20-30\% fall in the C-range. 
I reserve grades below a C-minus for those cases in which a student fails to attain a minimum level of basic competence in statistics, an absolute rather than relative standard. 
If you are in danger of failing to meet this minimum standard, you will receive a course problem notice.
I will only curve the course in your favor, so the most stringent possible grade boundaries are: A-range = 90-100, B-range = 80-89, C-range = 70-79, D-range = 60-69.
(In this case, the top two points of each range would be a ``plus'' and the bottom two points a ``minus.'')

\medskip


\noindent \textbf{Quizzes:} 
Your RIs will administer a number of short quizzes in recitation over the course of the semester: dates appear on the semester calendar on the \href{http://ditraglia.com/Econ103Public}{course website}.
Each quiz will cover basic material from the most recent lectures since the last quiz or midterm. 
When calculating your quiz average, I will drop your two lowest scores and weight the remaining quizzes evenly. 
Makeup quizzes will be given only in exceptional circumstances, so use your two ``free skips'' carefully.
Quizzes will not be returned and answers will not be posted but your RI will go over each quiz in recitation.
You are also welcome to view your quiz at your RI's office hours.

\medskip

\noindent \textbf{Exams:} 
There will be two 70-minute in-class midterm exams and a 2-hour final exam during the exam period.
Dates, times and locations will appear on the semester calendar on the \href{http://ditraglia.com/Econ103Public}{course website}.
Each midterm is worth 20\% and the final is worth 30\% of your grade.
Neither midterm is comprehensive, but the final is: it will focus on the final third of the course but include several questions on earlier material.
To give you a sense of the style and level of difficulty to expect, I have posted all of my past exams with full solutions on the \href{http://ditraglia.com/Econ103Public}{course website}.
There will be no makeup midterms: if you miss one midterm, your final exam will be worth 50\% to compensate; if you miss two midterms, it will be worth 70\%.
The makeup final will take place at the beginning of next semester and is outside of the instructor's control: eligibility as well as the time and date are determined by the Economics Department. 
Cheat sheets are not permitted on exams.
Scientific calculators are allowed but graphing calculators are not. 
You may write in pencil or pen on your exam as it will be photocopied before being returned to you.
We will check ID cards at each exam.

\medskip

\noindent \textbf{Regrade Requests:}
Exam regrade requests must be made in writing within a week of receiving your graded exam. 
As we re-grade the entire exam, your score could rise or fall. 
You may not discuss your answers with an RI or the instructor before submitting a regrade request. 


\medskip

\noindent \textbf{Extra Credit:} 
You can earn extra credit worth up to 5\% of your course grade for active in-class and online participation. 
Participation will be added to your overall course score after averaging all course assignments.
For example, if your overall weighted average on exams and quizzes is 83\%, your final score after factoring in extra credit will be between 83\% and 88\%, depending on how much extra credit you earn. 
Extra credit will be awarded for constructive questions, answers, and notes on Piazza, as well as for attendance and participation in in-class polls and activities.
Even if you post anonymously on Piazza, we can still award you extra credit for online participation.
Extra credit is discretionary and will calculated at the end of the semester by the course instructor in consultation with your RI.
As such, there is no precise formula that we can provide you in advance.
Extra credit is intended as a modest additional reward for students who are taking the course seriously, regardless of whether they struggle or excel on quizzes and exams. 
Attempts to game the system are not worth the effort: use your time to study instead.

\section*{Installing R and RStudio} First, download and install R from \url{http://cran.r-project.org/}. Second, download and install RStudio by visiting \url{http://rstudio.org/download/desktop} and clicking the link listed under ``Recommended for Your System.'' 
If you have trouble, ask your RI or the instructor for help in office hours.
Here are links to some additional free resources to help you learn R:
\begin{itemize}
		       \item \url{http://cran.r-project.org/other-docs.html}
\item \url{http://www.twotorials.com/}
			\item \url{http://www.r-bloggers.com/google-developers-r-programming-video-lectures/}
		 	\item \url{http://cran.r-project.org/doc/contrib/Farnsworth-EconometricsInR.pdf}
 			\item \url{http://www.ats.ucla.edu/stat/R/}
\end{itemize}

\end{document}
