\documentclass[11pt, letterpaper]{article}
\usepackage{geometry}
\geometry{margin=1in} 
\usepackage{totcount}
\usepackage{termcal}
\usepackage[T1]{fontenc}
\usepackage[sc]{mathpazo} 
\usepackage{setspace}
\linespread{1}
\usepackage{hyperref}
\usepackage{enumerate}
\usepackage{fancybox}
\usepackage{amsmath, amssymb}



%%%%%%%%%%%%% A Command to create an automatically numbered Quiz Icon in the course calendar.
\newtotcounter{quizcounter}
\newcommand{\quiz}{\addtocounter{quizcounter}{1}\fbox{\emph{Quiz \#\arabic{quizcounter}}}}

%%% HEADERS & FOOTERS
\usepackage{fancyhdr} 
\pagestyle{fancy} % options: empty , plain , fancy
\renewcommand{\headrulewidth}{0.4pt} % customise the layout...
%\chead{}\rhead{}
%\lfoot{}\cfoot{\thepage}\rfoot{}
%\rfoot{\footnotesize Econ 103}
\rhead{\footnotesize DiTraglia -- Spring 2014}
\lhead{\footnotesize Econ 103 Syllabus}
\renewcommand\footrulewidth{0pt}



% Some useful commands (our classes always meet either on Monday and Wednesday 
% or on Tuesday and Thursday)
\newcommand{\MWClass}{%
\calday[Monday]{\classday} % Monday
\skipday % Tuesday (no class)
\calday[Wednesday]{\classday} % Wednesday
\skipday % Thursday (no class)
\skipday % Friday 
\skipday\skipday % weekend (no class)
}

\newcommand{\TRClass}{%
\skipday % Monday (no class)
\calday[Tuesday]{\classday} % Tuesday
\skipday % Wednesday (no class)
\calday[Thursday]{\classday} % Thursday
\skipday % Friday 
\skipday\skipday % weekend (no class)
}

\newcommand{\Holiday}[2]{%
\options{#1}{\noclassday}
\caltext{#1}{#2}
}



\begin{document}


\thispagestyle{plain}

\begin{center}
\Large
\sc
Statistics for Economists\\
\large
Economics 103\\
\large
Spring 2014\\
\end{center}


\normalsize

\noindent \textbf{Instructor:} Francis DiTraglia \\
Office: 535 McNeil Building\\
Office Hours: M 8-10am

\bigskip
 
\noindent \textbf{Time and Location:} TR 9-10:30am, COLL 200  \\
% ECON-103  STAT FOR ECONOMISTS               1 CU
%      REGISTRATION REQUIRED FOR LEC, REC
%      COLLEGE QUANTITATIVE DATA ANALYSIS REQ.
%  001 LEC TR 9-10:30AM COLL 200       DITRAGLIA F
%      MAX: 115               
 
% ECON-103  RECITATION                        0 CU
%  201 REC M 10-11AM MCNB 169                QIU Z
%      MAX: 24                
%  202 REC F 9-10AM MCNB 410            KANTENGA K
%      MAX: 24                
%  203 REC M 12-1PM CAST A8             BAUGHMAN G
%      MAX: 24                
%  204 REC F 10-11AM WILL 215              ZHANG W
%      MAX: 24                
%  205 REC M 11-12NOON BENN 138         BAUGHMAN G
%      MAX: 24                
 
\noindent \textbf{Course Website (Canvas):} \url{https://upenn.instructure.com}\\
\noindent \textbf{Course Email:} \href{mailto:econ103@ditraglia.com}{\texttt{econ103@ditraglia.com}}
\medskip

\noindent \textbf{Recitations:}
\medskip \noindent

\begin{tabular}{lllll}
	Section 201 & Section 202 & Section 203 & Section 204 & Section 205\\
	MCNB 169 & MCNB 410 & CAST A8 & WILL 215 & BENN 138\\ 
	M 10--11am & F 9--10am & M 12--1pm & F 10--11am & M 11--noon
\end{tabular}

\medskip

\noindent \textbf{Recitation Instructors and Office Hours:}
{\footnotesize (*\emph{Head RI})}

\medskip
\noindent


\begin{tabular}{llll}
Garth Baughman*&Kory Kantenga& Zhesheng Qiu&Weilong Zhang\\
421 McNeil&	453 McNeil&477 McNeil & 312 McNeil\\
M 2--4pm&F 11--noon, 1--2pm & M 3:30--5:30pm& T 10--noon\\
\end{tabular}


\bigskip

\noindent \textbf{Course Description:} 
This course will teach you how to learn from data and understand uncertainty using the ideas of probability theory and statistics. 
After completing this course you will be able to carry out simple statistical analyses of your own using the computer package R.


\medskip

\noindent \textbf{Prerequisites:} T
he prerequisite for this course is multivariate calculus (Math 104 followed by 114 or 115). 
To do well in this course you will need to be comfortable with algebra, manipulating sums, differentiation and partial differentiation, solving unconstrained optimization problems, and integration. 
To help you gauge your level of mathematical preparation, the RIs will administer a math diagnostic quiz in the first recitation. 
This will not count towards your grade.




\medskip

\noindent \textbf{Required Text:} 
The required textbook for this course is \emph{Introductory Statistics for Business and Economics}, 4\textsuperscript{th} Edition by Thomas H.\ and Ronald J.\ Wonnacott (WW4). 
Used copies are readily available on \href{http://tinyurl.com/ECON103-2013A}{Amazon}. 
I strongly recommend that you do the assigned readings from this text, but my lecture slides are the final authority on course material. 
These will be posted online after each lecture. You are \emph{not} responsible for material in the textbook \emph{unless} it is also covered in lecture, but you \emph{are} responsible for material from lecture even if it is \emph{not} covered in the textbook.

\medskip


\noindent \textbf{Required Technology: } 
We will be using the ResponseCard NXT ``clicker'' for experiments and class participation exercises during the semester. You can buy or rent a clicker from the Penn Bookstore, but please note that models other than the NXT will \emph{not} work for this course. 
As clicker participation will make up 5\% of your course grade it is important that you bring your clicker to each lecture. 
I fully understand, however, that things can go wrong: your clicker might stop working or you might forget to bring it after pulling an all-nighter.
For this reason you will \emph{automatically} be excused from clicker participation for four lectures: there is no need to inform us in advance or after the fact. There will, however, be no further exceptions. 
I will begin keeping track of clicker participation in our second lecture. For more details, see ``Participation'' below. 
Because clickers will determine a portion of your grade, their use is subject to the code of academic integrity, as explained below under ``Academic Integrity.'' 

\newpage

\noindent \textbf{Required Software:} 
We will use the statistical package R via a front-end called RStudio throughout the course. 
Both programs are free and open source. See the last page of this syllabus for instructions on how to configure your computer to run R and RStudio. Both programs are also available in the Undergraduate Data Analysis Lab (UDAL) in McNeil rooms 104 and 108--9. 
You will be taught to use R primarily through a series of tutorials that I will assign as homework. (See ``Homework'' below.)  
Additional R resources are listed on the last page of this syllabus.

\bigskip

\noindent \textbf{Recommended Texts:} 
I recommend two supplementary texts for students who feel they may need extra help with the course material. 
The first is the \emph{Student Workbook to accompany Introductory Statistics for Business and Economics 4\textsuperscript{th} Edition}. 
Used copies are available on \href{http://www.amazon.com/gp/offer-listing/0471508993/sr=/qid=/ref=olp_page_2?ie=UTF8&colid=&coliid=&condition=all&me=&qid=&shipPromoFilter=0&sort=sip&sr=&startIndex=10}{Amazon}. 
The workbook contains full solutions to all odd-numbered problems from the textbook, while the text itself provides answers but no explanations. 
The workbook also contains extra practice problems with solutions. The second recommended text is \emph{The R Student Companion} by Brian Dennis.
This text is intended for those students who are having trouble learning R and prefer a physical book to the free online resources listed at the end of this document. 

\bigskip

\noindent \textbf{Lecture Recordings: } 
Audio and screen captures of all lectures will be automatically recorded and posted on \href{http://upenn.instructure.com}{Canvas}. 
This is a great way to get caught up if you miss a lecture.

\bigskip

\noindent \textbf{Departmental Course Policies: } 
All Economics Department course policies are in force in Econ 103 even if not explicitly listed on this syllabus. 
See: \url{http://economics.sas.upenn.edu/undergraduate-program/course-information/guidelines/policies} for full details. 


\bigskip


\noindent \textbf{Academic Integrity: } 
All suspected violations of the code of academic integrity as set forth in the Pennbook will be reported to the Office of Student Conduct. 
Confirmed violations will result in failure for the course. 
Because it will be used to determine your class participation grade, operating a clicker on behalf of another student is cheating. 
If you are discovered using a clicker other than your own or have votes in a class that you did not attend, you will face the penalties described above. 
We will check identification cards at exams so please be sure to bring yours.

\bigskip

\noindent \textbf{Piazza:} 
We will be using an online discussion forum called Piazza for this course, which you can access directly from \href{http://upenn.instructure.com}{Canvas}. 
Piazza is where we will make all course announcements, assign homework and readings, post lecture slides, exam practice problems, and solutions throughout the semester. 
Piazza is also the best place to ask any question you have about course material or logistics. 
By asking your question and getting an answer on Piazza, you create a positive externality: other students benefit from your questions and you benefit from theirs. 
The instructor and RIs will actively moderate Piazza both to answer questions and approve (or correct) answers written by your fellow-students.
As an incentive, I will award ``free points'' worth 5\% of your grade for making active use of the forum. 
See ``Participation'' below for more details.

\bigskip

\noindent \textbf{Attendance:} 
Regularly attending lectures is the only way to earn clicker participation points. 
As described above under ``Required Technology'' you will \emph{automatically} be excused from clicker participation for four lectures, but there will be no further exemptions of any kind. 
Similarly, regularly attending recitations is the only way to avoid a string of zeros on the quizzes (see ``Quizzes'' below). 
I will drop your two lowest quiz grades, including absences.

\bigskip

\noindent \textbf{Email Policy:} 
Email is for personal issues only. Questions about course material or logistics should be reserved for Piazza, lectures, recitations, and office hours. 
When contacting the instructor or your RI please use the course email address \href{mailto:econ103@ditraglia.com}{\texttt{econ103@ditraglia.com}} rather than our personal addresses to ensure that you receive a response.

\newpage

\section*{Assignments and Grading}
	\begin{equation*}
	\begin{split}
		\mbox{Final Grade} = (&5\% \times \mbox{Clicker Participation}) + (5\% \times \mbox{Piazza Participation}) + (20\% \times \mbox{Quizzes})  \\ &
							 + (20\% \times \mbox{Midterm 1}) + (20\% \times \mbox{Midterm 2}) + (30\% \times \mbox{Final})
	\end{split}
	\end{equation*}
If necessary, I will curve final course grades (\emph{not} individual assignments) in accordance with Departmental guidelines: approximately 20-30\% in the A-range, 40-50\% in the B-range, and the bulk of the remaining 20-40\% in the C-range. 
I reserve grades below a C-minus for those rare cases in which a student fails to attain a minimum level of basic competence in statistics, an absolute rather than relative standard. 
Ideally no one will fall below this level. 
If I see that you are in danger of failing to meet this minimum standard, I will contact you to set up a meeting. 
As I will \emph{only} curve in your favor, the most stringent possible grade boundaries are: A-range = 90-100, B-range = 80-89, C-range = 70-79, D-range = 60-69, F = below 60. 
In this case top two points of each range would correspond to a ``plus'' and the bottom two to a ``minus.'' 

\medskip


\noindent \textbf{Clicker Participation:} 
Each lecture will feature activities in which you can earn participation credit by voting with your clicker. 
If you attend a given lecture and participate in the majority of the clicker activities, you will be counted as ``present.'' 
If you are ``present'' at at least 80\% of the semester's lectures, I will award you 100\% for clicker participation; otherwise I will deduct points proportionally. 
Hence, you are \emph{automatically} excused from two full weeks of lectures. 
This includes absences and forgotten or malfunctioning clickers. There will be no further exceptions.


\medskip

\noindent \textbf{Piazza Participation:} 
You will earn further participation credit based on the frequency and quality of your contributions on Piazza, including questions, answers, and follow-ups. 
If you participate actively, you will receive 100\%: these are essentially ``free points.''  
You \emph{must} contribute to earn points, but spamming the boards with unhelpful contributions will not gain you credit.  


\medskip



\noindent \textbf{Homework:} 
I will post a number of Problem Sets and R Tutorials on Piazza with approximate ``due dates.'' 
Although homework will neither be collected nor graded, it is \emph{extremely important}: it is impossible to do well on exams without completing the homework regularly, including R Tutorials. 
I will post solutions after each ``due date'' except for questions with an answer in the back of the textbook. 
Use these responsibly: you gain nothing from simply reading the solutions.


 \medskip

\noindent \textbf{Quizzes:} 
Your RIs will administer \total{quizcounter} short quizzes in recitation over the course of the semester. 
(See the semester calendar below.) 
Each quiz will cover basic material from the most recent lectures since the last quiz or midterm. 
When calculating your quiz average, I will drop your two lowest scores and weight the remaining quizzes evenly. 
No makeup quizzes will be given so use your two ``free skips'' carefully.
Quizzes will not be returned and answers will not be posted.

\medskip

\noindent \textbf{Exams:} 
There will be two in-class midterm exams, February 11\textsuperscript{th} and March 25\textsuperscript{th}, and a final on May 7\textsuperscript{th}.
Each midterm is worth 20\% and the final is worth 30\% of your grade. 
All exams are comprehensive but will focus on material not covered on previous exams. 
Attendance at all exams is \emph{mandatory} and there will be no makeup midterms. 
In exceptional circumstances, e.g.\ a death in the family or a serious documented illness, please contact the instructor in advance. 
The makeup final will take place in Spring 2014 and is outside of the instructor's control: eligibility as well as the time and date are determined by the Economics Department. 
Regrade requests must be made in writing within a week of receiving your graded exam. 
As we re-grade the entire exam, your score could rise or fall. 
You may not discuss your answers with an RI or the instructor before submitting a regrade request. 
Exams will be photocopied before being returned and you may write in pencil or pen. 
Scientific calculators are permitted but graphing calculators are not. 
We will check ID cards at each exam so please be sure to bring yours. 



\newpage


%NOTE: don't use leading zeros in dates! In other words, use 1/1/2014 rather than 01/01/2014

\begin{center}
\small
\begin{calendar}{1/13/2014}{16} %Date of Monday in first week of classes, NOT the date of the first class!
\setlength{\calboxdepth}{.25in}
\TRClass
% schedule
\caltexton{1}{Introduction \hfill \fbox{\emph{No Recitations}}}
\caltextnext{Descriptive Stats.\ \& Graphics I }
\caltextnext{Descriptive Stats.\ \& Graphics II \\ \vspace{0.25em}\hfill \fbox{Recitation: Math Diagnostic}}
\caltextnext{Regression I}
\caltextnext{Basic Probability I \hfill\quiz}
\caltextnext{Basic Probability II}
\caltextnext{Basic Probability III\\ \hfill \fbox{\emph{Recitation: Midterm Review}}}
\caltextnext{Discrete RVs I}
\caltextnext{Discrete RVs II  \hfill\quiz}
\caltextnext{Discrete RVs III}
\caltextnext{Discrete RVs IV, Continuous RVs I \hfill \quiz}
\caltextnext{Continuous RVs II}
\caltextnext{Continuous RVs III \hfill \fbox{\emph{No Recitations}}}
\caltextnext{Sampling Dists.\ \& Estimation I}
\caltextnext{Sampling Dists.\ \& Estimation II \\ \vspace{0.25em} \hfill \fbox{\emph{Recitation: Midterm Review}}}
\caltextnext{Confidence Intervals I}
\caltextnext{Confidence Intervals II \hfill\quiz}
\caltextnext{Confidence Intervals III}
\caltextnext{Confidence Intervals IV \hfill\quiz}
\caltextnext{Hypothesis Testing I}
\caltextnext{Hypothesis Testing II \hfill \quiz}
\caltextnext{Hypothesis Testing III}
\caltextnext{Regression II \hfill \quiz}
\caltextnext{Regression III}
% ... and so on

% Holidays
\Holiday{1/14/2014}{\textbf{Winter Break -- No Class}}
\Holiday{2/13/2014}{\textbf{Snow Day!}}
\Holiday{3/11/2014}{\textbf{Spring Break -- No Class}}
\Holiday{3/13/2014}{\textbf{Spring Break -- No Class}\hfill \fbox{\emph{No Recitations}}}
\Holiday{4/1/2014}{\textbf{Class Cancelled}}

%\Holiday{9/2/2013}{\textbf{Labor Day -- No Class}}
%\Holiday{9/18/2013}{\textbf{Instructor in Amsterdam -- No Class}\\\hfill %\quiz}
%\Holiday{11/27/2013}{\textbf{No Econ 103 -- Go to Friday Classes}\\\hfill \fbox{\emph{No Recitations This Week}}}
% ... and so on


\options{2/11/2014}{\noclassday} % first midterm exam
\caltext{2/11/2014}{\shadowbox{\textbf{Midterm I} -- Material through Feb.\ 7th}}
\options{3/25/2014}{\noclassday} %second midterm exam
\caltext{3/25/2014}{\shadowbox{\textbf{Midterm II} -- Material through March\ 20th}}
\options{5/1/2014}{\noclassday} % finals
\caltext{5/1/2014}{\textbf{Reading Day -- No Class}}
%\options{12/19/2013}{\noclassday} % finals
%\caltext{12/19/2013}{\doublebox{\textbf{Final} (6--8pm, Location TBA)}}
\end{calendar}
\end{center}

            
\section*{Installing R and RStudio} First, download and install R from \url{http://cran.r-project.org/}. Second, download and install RStudio by visiting \url{http://rstudio.org/download/desktop} and clicking the link listed under ``Recommended for Your System.'' 

\bigskip

\section*{Additional R Resources} While not required, these references may be useful if you need some extra help learning R, or want to go beyond the material covered in the course.
\begin{itemize}
		       \item Contributed Documentation -- Comprehensive R Archive Network (CRAN) \\\url{http://cran.r-project.org/other-docs.html}
           	\begin{quote}
           		Comprehensive list of freely available reference material for R.
           	\end{quote}
\item R Twotorials -- Anthony Damico \\\url{http://www.twotorials.com/}
		\begin{quote}
		Ninety energetic, two-minute video tutorials on statistical programming with R. 
		\end{quote}
			\item Google Developers R Programming Video Lectures\\ \url{http://www.r-bloggers.com/google-developers-r-programming-video-lectures/}\begin{quote}R Programming video tutorials from beginning to advanced. \end{quote}
		 	\item Econometrics in R -- Grant Farnsworth\\\url{http://cran.r-project.org/doc/contrib/Farnsworth-EconometricsInR.pdf}
 		\begin{quote}
 		If you'd like to keep using R in Econ 104, this is what you should read.
 		\end{quote}
 			\item Resources to help you learn R -- UCLA Academic Technology Services \\\url{http://www.ats.ucla.edu/stat/R/}
		\begin{quote}
			A wealth of information about R, conveniently arranged in one place. The ``R Starter Kit'' is particularly helpful.
		\end{quote}
	           \item R in a Nutshell -- Adler\\ \url{http://proquestcombo.safaribooksonline.com/book/programming/r/9781449377502}         
           	\begin{quote}
           		Electronic version of the book of the same name published by O'Reilly (Accessible on the UPenn Network). Provides a comprehensive reference guide to R.
           	\end{quote}
		\item R-bloggers \\\url{http://www.r-bloggers.com}
		\begin{quote}
			A blog aggregator for R news and tutorials, with lots of applications.
		\end{quote}
\end{itemize}

% \newpage

% \thispagestyle{empty}
% \begin{center}
% 	[This page intentionally left blank]
% \end{center}


% \newpage

% \thispagestyle{empty}
% \begin{center}
% 	[This page intentionally left blank]
% \end{center}


% \newpage


% \thispagestyle{empty}

% \Large



% \begin{center}
%   \fbox{\fbox{\parbox{5.5in}{\centering
%         I have read and understood the syllabus for Econ 103 and agree to abide by all course policies contained therein.}}}
% \end{center}
% \vspace{0.2in}
% \makebox[\textwidth]{Name:\enspace\hrulefill}

% \vspace{0.2in}
% \noindent \makebox[\textwidth]{Student ID \#:\enspace\hrulefill}

% \vspace{0.3in}
% \noindent\makebox[\textwidth]{Signature:\enspace\hrulefill}
                                                                                                                                                                                                               
\end{document}



